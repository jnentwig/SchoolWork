\usepackage[toc,xindy]{glossaries}

\makenoidxglossaries

\newglossaryentry{set}
{
	name={set},
	description={A collection of objects}
}
\newglossaryentry{vertex}
{
	name={vertex},
	description={A point or node in a graph},
	plural={verticies}
}
\newglossaryentry{edge}
{
	name={edge},
	description={A connection or line between points in a graph}
}
\newglossaryentry{connected}
{
	name={connected},
	description={A graph where one may start at any point and could follow edges and eventually get to any other point}
}
\newglossaryentry{multigraph}
{
	name={multigraph},
	description={Like a graph, but multiple \glspl{edge} may connect the same \gls{vertex} pair, and an edge may connect a \glx{vertex} to itself}
}
\newglossaryentry{complete}
{
	name={complete graph},
	description={A graph with all possible \glspl{edge} included, the notation $k_n$ is used to denote the \gls{complete} with $n$ \glspl{vertex}}
}
\newglossaryentry{face}
{
	name={face},
	description={An section of the space we are embedding our graph in that is separated from the rest of the space by \glspl{edge}}
}
\newglossaryentry{contraction}
{
	name={contraction},
	description={A graph operation where one removes an edge by fusing two \glspl{vertex} together}
}
\newglossaryentry{subgraph}
{
	name={subgraph},
	description={$A$ is a \gls{subgraph} of $B$ if $A$'s \gls{vertex} \gls{set} and \gls{edge} \gls{set} are subsets of $B$'s \gls{vertex} \gls{set} and \gls{edge} \gls{set} respectively}
}