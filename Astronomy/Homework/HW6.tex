\documentclass{article}
\usepackage{amsmath,amssymb,graphicx,algpseudocode,algorithm,amsthm}
\usepackage[margin=1in]{geometry}
\usepackage{mathrsfs}
\let\mathcrl\mathscr
\usepackage[mathscr]{euscript}
\usepackage{marginnote}
\usepackage{hyperref}
\usepackage{qtree}
\usepackage{graphicx}
\usepackage{tikz}
\geometry{reversemarginpar}


\author{Benji Altman}

\def\latex{\LaTeX\ }

\def\useLim{\limits}
\newcommand{\question}[1]{\marginnote{#1}}
\let\union\cup
\let\inter\cap
\let\emptyset\varnothing
\let\bigunion\bigcup
\let\biginter\bigcap
\let\composed\circ
\let\cross\times
\def\And{\textit{ and }}
\def\Or{\textit{ or }}
\def\sbSeperator{\,\middle|\,}
\def\Return{\State\textbf{return}\par}
\def\ZNonNegative{{\mathbb Z_{\ge 0}}}
\newcommand{\setcomp}[1]{{#1}^{\mathsf{c}}}
\newcommand{\prodfrom}[3]{\prod\useLim_{#1}^{#2}\LB {#3} \RB}
\newcommand{\sumfrom}[3]{\sum\useLim_{#1}^{#2} \LB {#3} \RB}
\newcommand{\unionfrom}[3]{\bigunion\useLim_{#1}^{#2} \LB {#3} \RB}
\newcommand{\interfrom}[3]{\biginter\useLim_{#1}^{#2} \LB {#3} \RB}
\newcommand{\interacross}[2]{\interfrom{#1}{}{#2}}
\newcommand{\unionacross}[2]{\unionfrom{#1}{}{#2}}
\newcommand{\sumacross}[2]{\sumfrom{#1}{}{#2}}
\newcommand{\prodacross}[2]{\prodfrom{#1}{}{#2}}
\newcommand{\Lim}[3]{\lim\useLim_{{#1} \to {#2}}\LB {#3} \RB}
\newcommand{\set}[1]{\left\{ {#1} \right\}}
\newcommand{\setbuilder}[2]{\left\{{#1} \sbSeperator {#2}\right\}}
\newcommand{\derivative}[2]{\frac{d}{d{#2}}\LB {#1} \RB}
\newcommand{\Exists}[2]{\exists_{#1}\LB {#2} \RB}
\newcommand{\All}[2]{\forall_{#1}\LB {#2} \RB}
\newcommand{\abs}[1]{\left|{#1}\right|}
\newcommand{\card}[1]{\left| {#1} \right|}
\newcommand{\range}[1]{\textit{\textbf{Rng}}\left( {#1} \right)}
\newcommand{\domain}[1]{\textit{\textbf{Dom}}\left( {#1} \right)}
\newcommand{\pset}[1]{\mathcal P\left( {#1} \right)}
\newcommand{\pair}[2]{\left( {#1} , {#2} \right)}
\def\closure{\overline}
\newcommand{\limpts}[1]{{#1} '}
\newcommand{\ooint}[2]{\left( {#1} , {#2} \right)}
\newcommand{\ocint}[2]{\left( {#1} , {#2} \right]}
\newcommand{\coint}[2]{\left[ {#1} , {#2} \right)}
\newcommand{\ccint}[2]{\left[ {#1} , {#2} \right]}
\newcommand{\eqclass}[1]{\bar{#1}}
\newcommand{\ceil}[1]{\left\lceil {#1} \right\rceil}
\newcommand{\floor}[1]{\left\lfloor {#1} \right\rfloor}
\newcommand{\inv}[1]{{#1}^{-1}}
\def\true{\text{True}}
\def\false{\text{False}}
\newcommand{\ball}[2]{B_{#1}\left({#2}\right)}
\def\LB{}
\def\RB{}
\newcommand{\cannonicalSet}[1]{\left[ #1 \right]}
\let\lxor\oplus
\newcommand{\norm}[1]{\left|\left|{#1}\right|\right|}

\newtheorem{theorem}{Theorem}[section]
\newtheorem{lemma}[theorem]{Lemma}
\theoremstyle{definition}
\newtheorem{definition}{Definition}[section]


\usepackage{ wasysym }
\usepackage{enumitem}

\title{Astronomy\\Homework 6}

\begin{document}
	\maketitle
	
	\section{Chapter 13}
	\question{1.} A large cloud of $H_2$ (molecular hydrogen) where stars are able to form. Protostars will form when gas in a molecular cloud clumps together due to gravity and this clump is the protostar, eventually it will get enough pressure to start fusion and become a star.
	
	\question {2.} Once the sun runs out of hydrogen to fuse it will start to fuse heavier elements like helium, this will give off more energy, causing the sun to expand into a red giant. Hydrogen shell fusion refers to hydrogen around the core of the sun being fused into helium. Helium fusion refers to fusing helium atoms into heavier elements. Double shell fusion refers to when helium is fused in an inner shell and hydrogen is fused around that in an outer shell.
	
	\question{3.} A high mass star burns through it's supply of hydrogen much faster (and brighter) then a low mass star. Thus high mass stars die quickly and low mass stars die slowly. A high mass star may go out in a spectacular super-nova, whereas low mass stars aren't quite as flashy.
	
	\question{4.} Iron when fused together creates an atom that has an atomic mass more then twice Iron's, thus some energy must be lost in the fusion.
	
	\question{5.} Let $r = 1 \text{ AU}$, thus $V = \frac43\pi r^3 = \frac43\pi\text{ AU}^3 \approx 4.189 \text{ AU}^3 \approx 1.402\cdot10^{34}\text{ m}^3$ where $V$ is the volume of a sphere with radius $1 \text{ AU}$. The density of this sphere, if it has mass $m = 1\text{ M}_{\astrosun}$, would then be $d = \frac mV \approx 1.418\cdot 10^{-4} \text{ kg}\text{ m}^{-3}$, which is less dense then water and the atmosphere at sea level on earth which are $1000 \text{ kg}\text{ m}^{-3}$ and $1\text{ kg}\text{ m}^{-3}$ respectively. 
	
	\question{6.} Let us define $\circ$ to be the `fuse with' operator
	\begin{enumerate}[label=\alph*)]
		\item Magnesium
		\item Sulfur
		\item Nickel
	\end{enumerate}
	
\end{document}