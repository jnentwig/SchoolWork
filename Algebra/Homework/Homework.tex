\documentclass{article}

\usepackage{amsmath,amssymb,graphicx,algpseudocode,algorithm,amsthm}
\usepackage[margin=1in]{geometry}
\usepackage{mathrsfs}
\let\mathcrl\mathscr
\usepackage[mathscr]{euscript}
\usepackage{marginnote}
\usepackage{hyperref}
\usepackage{qtree}
\usepackage{graphicx}
\usepackage{tikz}
\geometry{reversemarginpar}


\author{Benji Altman}

\def\latex{\LaTeX\ }

\def\useLim{\limits}
\newcommand{\question}[1]{\marginnote{#1}}
\let\union\cup
\let\inter\cap
\let\emptyset\varnothing
\let\bigunion\bigcup
\let\biginter\bigcap
\let\composed\circ
\let\cross\times
\def\And{\textit{ and }}
\def\Or{\textit{ or }}
\def\sbSeperator{\,\middle|\,}
\def\Return{\State\textbf{return}\par}
\def\ZNonNegative{{\mathbb Z_{\ge 0}}}
\newcommand{\setcomp}[1]{{#1}^{\mathsf{c}}}
\newcommand{\prodfrom}[3]{\prod\useLim_{#1}^{#2}\LB {#3} \RB}
\newcommand{\sumfrom}[3]{\sum\useLim_{#1}^{#2} \LB {#3} \RB}
\newcommand{\unionfrom}[3]{\bigunion\useLim_{#1}^{#2} \LB {#3} \RB}
\newcommand{\interfrom}[3]{\biginter\useLim_{#1}^{#2} \LB {#3} \RB}
\newcommand{\interacross}[2]{\interfrom{#1}{}{#2}}
\newcommand{\unionacross}[2]{\unionfrom{#1}{}{#2}}
\newcommand{\sumacross}[2]{\sumfrom{#1}{}{#2}}
\newcommand{\prodacross}[2]{\prodfrom{#1}{}{#2}}
\newcommand{\Lim}[3]{\lim\useLim_{{#1} \to {#2}}\LB {#3} \RB}
\newcommand{\set}[1]{\left\{ {#1} \right\}}
\newcommand{\setbuilder}[2]{\left\{{#1} \sbSeperator {#2}\right\}}
\newcommand{\derivative}[2]{\frac{d}{d{#2}}\LB {#1} \RB}
\newcommand{\Exists}[2]{\exists_{#1}\LB {#2} \RB}
\newcommand{\All}[2]{\forall_{#1}\LB {#2} \RB}
\newcommand{\abs}[1]{\left|{#1}\right|}
\newcommand{\card}[1]{\left| {#1} \right|}
\newcommand{\range}[1]{\textit{\textbf{Rng}}\left( {#1} \right)}
\newcommand{\domain}[1]{\textit{\textbf{Dom}}\left( {#1} \right)}
\newcommand{\pset}[1]{\mathcal P\left( {#1} \right)}
\newcommand{\pair}[2]{\left( {#1} , {#2} \right)}
\def\closure{\overline}
\newcommand{\limpts}[1]{{#1} '}
\newcommand{\ooint}[2]{\left( {#1} , {#2} \right)}
\newcommand{\ocint}[2]{\left( {#1} , {#2} \right]}
\newcommand{\coint}[2]{\left[ {#1} , {#2} \right)}
\newcommand{\ccint}[2]{\left[ {#1} , {#2} \right]}
\newcommand{\eqclass}[1]{\bar{#1}}
\newcommand{\ceil}[1]{\left\lceil {#1} \right\rceil}
\newcommand{\floor}[1]{\left\lfloor {#1} \right\rfloor}
\newcommand{\inv}[1]{{#1}^{-1}}
\def\true{\text{True}}
\def\false{\text{False}}
\newcommand{\ball}[2]{B_{#1}\left({#2}\right)}
\def\LB{}
\def\RB{}
\newcommand{\cannonicalSet}[1]{\left[ #1 \right]}
\let\lxor\oplus
\newcommand{\norm}[1]{\left|\left|{#1}\right|\right|}

\newtheorem{theorem}{Theorem}[section]
\newtheorem{lemma}[theorem]{Lemma}
\theoremstyle{definition}
\newtheorem{definition}{Definition}[section]

\let\setminus-
\let\LB[
\let\RB]
\renewcommand\setcomp[1]{{#1}'}
\renewcommand\question[1]{\marginnote{\textbf{#1}}}
\renewcommand\eqclass[1]{\left[{#1}\right]}


\title{Algebra Homework}


\begin{document}
\maketitle
\tableofcontents

\section{Chapter 1}
\subsection{Section 1}
\subsubsection{Question 1}

Choose $a,b\in S$. We find $$a = a * b = b * a = b$$, and thus all elements in $S$ must be the same element, so there is most one element of $S$.

\subsubsection{Question 2}
Let us choose $a,b,c \in S$.

\question{(a)} We have $$a*b = a - b = -(b-a) = -(b*a)$$, thus iff $0 = a*b = a-b$ we have $a*b = b*a$ as $0=-0$, however for any other value of $a*b$, $a*b\not=b*a$. We also may notice that iff $a=b$, then $a*b=a-b=0$. Thus for all $a\not= b$, $a*b \not=b*a$.

\question{(b)} We have
\begin{align*}
a*(b*c) &= a-(b-c) \\
&= a+(c-b) \\
&= a+c-b \\
&= a-b+c \\
&= a-b-(-c) \\
&= (a-b)-(-c) \\
&= (a*b)*-c
\end{align*} so $a*(b*c) = (a*b)*c$ iff $c=-c$ which is only true if $c = 0$.

\question{(c)} We have $a*0 = a-0 = a$.

\question{(d)} We have $a*a = a-a = 0$.

\subsection{Section 2}

\subsubsection{Question 8}

Let $x \in (A\setminus B) \union (B\setminus A)$ then either $x \in A \setminus B$ or $x \in B\setminus A$. If $x \in A\setminus B$ then we get that $x \in A$ and $x \not\in B$, thus $x \in A\union B$ and $x \not\in A \inter B$, which would mean $x \in (A\union B)\setminus(A\inter B)$. If $x\in B\setminus A$ then we get that $x\in B$ and $x\not\in A$, thus $x\in A \union B$ and $x\not\in A\inter B$, which would mean $x \in (A\union B)\setminus(A\inter B)$. It has now been demonstrated that $(A\setminus B) \union (B\setminus A) \subset (A \union B) \setminus (B\inter A)$.

Now let $x\in(A\union B)\setminus(A\inter B)$. We have that $x \in A\union B$ and $x \not\in A\inter B$. It follows that either $x \in A$ or $x\in B$, however, $x$ is not in both $A$ and $B$. This may be written as: $x \in A$ and $x \not\in B$, or $x\in B$ and $x\not\in A$. This then translates to $x \in A\setminus B$ or $x\in B\setminus A$, therefore, $x\in (A\setminus B) \union (B\setminus A)$. It has now been demonstrated that $(A\union B) \setminus (B\inter A) \subset (A\setminus B)\union (B\setminus A)$.

Now it has been shown that both sets are subsets of each-other, thus $(A\setminus B)\union(B\setminus A) = (A\union B) \setminus (A\inter B)$.



This may be displayed pictorially as follows:

%TODO figure out pictures

\def\secondcircle{(210:0.95cm) circle (1.5cm)}
\def\thirdcircle{(330:0.95cm) circle (1.5cm)}
\begin{tikzpicture}
	\begin{scope}
		\clip \secondcircle;
		\fill[cyan] \thirdcircle;
	\end{scope}
	\draw \secondcircle node [text=black,below left] {$A$};
	\draw \thirdcircle node [text=black,below right] {$B$};
\end{tikzpicture}


\subsubsection{Question 9}

Let $x \in A \inter (B \union C)$, thus $x \in A$ and $x\in B\union C$. We then have that $x \in B$ or $x \in C$. Now as we already know that $x\in A$ then we get that either $x \in B \inter A$ or $x \in C \inter A$ and therefore $x \in (A\inter B) \union (A \inter C)$. Thus it has been shown that $A \inter (B\union C) \subset (A\inter B)\union (A\inter C)$.

Let $x \in (A\inter B) \union (A\inter C)$, thus $x \in (A\inter B)$ or $x\in (A\inter C)$. We then get that either $x \in A$ and $x \in B$ or that $x\in A$ and $x\in C$, either way $x\in A$, thus we may write that $x \in A$ and either $x\in B$ or $x\in C$. This would be the same as $x\in A$ and $x\in B\union C$, which then translates to $x \in A \inter (B\union C)$. Thus it has been shown that $(A\inter B)\union (A\inter C) \subset A\inter (B\union C)$.

We have now shown that both sets are subsets of each-other, thus $A\inter(B\union C) = (A\inter B)\union (A\inter C)$.

\subsubsection{Question 10}

Let $x\in A\union (B\inter C)$, assume then for the sake of contradiction that $x\not\in(A\union B)\inter (A\union C)$. Because $x \in A\union (B\inter C)$ we have that $x \in A$ or $x\in B\inter C$. Because $x\not\in (A\union B)\inter (A\union C)$ we have that $x\not\in A\union B$ or $x\not\in A\union C$. We then get that either $x \not\in A$ and $x\not\in B$ or $x\not\in A$ and $x\not\in C$, either way $x\not\in A$, so we have $x\in B\inter C$. We know that $x\not\in B$ or $x\not\in C$, however we also have that $x\in B$ and $x\in C$ due to $x\in B\inter C$, thus we have a contradiction. Thus $A\union(B\inter C) \subset (A\union B)\inter(A \union C)$.

Let $x\in (A \union B) \inter (A\union C)$ and assume for the sake of contradiction that $x\not\in A\union(B \inter C)$. We then get that $x\not\in A$ and $x\not\in B\inter C$. We also have that $x\in A \union B$ and $x\in A \union C$, so if $x\not\in A$ then we get $x\in B$ and $x\in C$. This is then translated to $x\in B\inter C$ which is a direct contradiction with $x\not\in B\inter C$ and again we have a contradiction. Thus $(A\union B)\inter(A\union C)\subset A\union (B\inter C)$.

We have now shown that both sets are subsets of each other, thus $A\inter(B\union C) = (A\union B)\inter(A\union C)$.

\subsubsection{Question 12}

\question{(a)}
\begin{align*}
	\setcomp{(A\union B)} &= \setbuilder{x\in S}{x\not\in A\union B} \\
	&= \setbuilder{x\in S}{x\not\in A \And x\not\in B} \\
	&= \setbuilder{x\in S}{x \in \setcomp A \And x\in \setcomp B} \\
	&= \setcomp A \inter \setcomp B
\end{align*}

\question{(b)}
\begin{align*}
	\setcomp{(A\inter B)} &= \setbuilder{x\in S}{x\not\in A\inter B} \\
	&= \setbuilder{x\in S}{x\not\in A \Or x\not\in B} \\
	&= \setbuilder{x\in S}{x \in \setcomp A \Or x\in \setcomp B} \\
	&= \setcomp A \union \setcomp B
\end{align*}

\subsubsection{Question 13}
\question{(a)}
\begin{align*}
	A + B &= (A\setminus B) \union (B\setminus A) \\
	&= (B\setminus A) \union (A\setminus B) \\
	&= B + A
\end{align*}

\question{(b)}
First notice that for any set $X$, $X\setminus \emptyset = A$ and that $\emptyset \setminus X = \emptyset$.
\begin{align*}
	A + \emptyset &= (A \setminus \emptyset) \union (\emptyset \setminus A)\\
	&= A \union \emptyset \\
	&= A \\
\end{align*}

\question{(c)}
\begin{align*}
	A\cdot A &= A\inter A\\
	&= A
\end{align*}

\question{(d)}
\begin{align*}
	A + A &= (A\setminus A)\union (A\setminus A) \\
	&= \emptyset \union \emptyset \\
	&= \emptyset
\end{align*}

\question{(e)} To simplify this question let me introduce the logical operation, $a \lxor b$ which is defined as either $a$ or $b$ but not both, and we will show that $a \lxor (b\lxor c) = (a\lxor b)\lxor c$ using truth tables.

\begin{center}
	\begin{tabular}{c|c|c|c|c|c|c}
		$a$&$b$&$c$&$a\lxor b$&$b\lxor c$&$a\lxor(b\lxor c)$&$(a\lxor b)\lxor c$ \\
		\false&\false&\false&\false&\false&\false&\false\\
		\false&\false&\true&\false&\true&\true&\true\\
		\false&\true&\false&\true&\true&\true&\true\\
		\false&\true&\true&\true&\false&\false&\false\\
		\true&\false&\false&\true&\false&\true&\true\\
		\true&\false&\true&\true&\true&\false&\false\\
		\true&\true&\false&\false&\true&\false&\false\\
		\true&\true&\true&\false&\false&\true&\true
	\end{tabular}
\end{center}

Now we wish to show that $A + B = \setbuilder{x\in S}{x \in A \lxor x \in B}$. To do this we will first show that $a\lxor b = (a\land\lnot b) \lor (b\land\lnot a)$, where $\lnot$ is a logical not, $\land$ is a logical and, and $\lor$ is a logical or. We again show this by the following truth table:
\begin{center}
	\begin{tabular}{c|c|c|c|c|c|c|c}
		$a$&$b$&$\lnot b$&$a\land\lnot b$&$\lnot a$&$b\land\lnot a$&$(a\land\lnot b)\lor(b\land\lnot a)$&$a\lxor b$\\
		\false&\false&\true&\false&\true&\false&\false&\false\\
		\false&\true&\false&\false&\true&\true&\true&\true\\
		\true&\false&\true&\true&\false&\false&\true&\true\\
		\true&\true&\false&\false&\false&\false&\false&\false\\
	\end{tabular}
\end{center}
Now we find
\begin{align*}
A + B &= \setbuilder{x\in S}{x\in A+B} \\
&= \setbuilder{x\in S}{x \in (A\setminus B)\union(B\setminus A)}\\
&= \setbuilder{x\in S}{x\in (A\setminus B) \lor x\in(B\setminus A)}\\
&=\setbuilder{x\in S}{(x\in A \land x\not\in B)\lor(x\in B\land x\not\in A)}\\
&=\setbuilder{x\in S}{x\in A\lxor x\in B}\\
\end{align*}
so we then have
\begin{align*}
A+(B+C)&=\setbuilder{x\in S}{x\in A\lxor x\in B+C} \\
&=\setbuilder{x\in S}{x\in A\lxor(x\in B\lxor x\in C)}\\
&=\setbuilder{x\in S}{(x\in A \lxor x\in B)\lxor x\in C}\\
&=\setbuilder{x\in S}{x \in A + B\lxor x \in C}\\
&=(A+B)+C
\end{align*}

\question{(f)}
Suppose $B\not= C$. Because $B\not= C$ there exists some $x\in S$ such that either $x\in B$ and $x\not\in C$ or $x\in C$ and $x\not\in B$, we will assume without loss of generality that $x\in B$ and $x\not\in C$. Now if $x\in A$ then we would find $x\not\in A+B$ and $x\in A+C$. If $x\not\in A$ we would find that $x\in A+B$ and $x\not\in A+C$. We now have shown that $B\not= C\implies A+B\not=A+C$, thus by contrapositive we have $A+B=A+C \implies B=C$.

\question{(g)}
First we will want to show logical equivalence between the statement $a\land(b\lxor c)$ and $(a\land b)\lxor(a\land c)$.
\begin{center}
	\begin{tabular}{c|c|c|c|c|c|c|c}
		$a$&$b$&$c$&$b\lxor c$&$a\land b$&$a\land c$&$a\land(b\lxor c)$&$(a\land b)\lxor(a\land c)$\\
		\false&\false&\false&\false&\false&\false&\false&\false\\
		\false&\false&\true&\true&\false&\false&\false&\false\\
		\false&\true&\false&\true&\false&\false&\false&\false\\
		\false&\true&\true&\false&\false&\false&\false&\false\\
		\true&\false&\false&\false&\false&\false&\false&\false\\
		\true&\false&\true&\true&\false&\true&\true&\true\\
		\true&\true&\false&\true&\true&\false&\true&\true\\
		\true&\true&\true&\false&\true&\true&\false&\false
	\end{tabular}
\end{center}
now we may show
\begin{align*}
A\cdot(B+C) &= A\inter(B+C)\\
&=\setbuilder{x\in S}{x \in A\inter(B+C)}\\
&= \setbuilder{x\in S}{x\in A\land x\in(B+C)} \\
&= \setbuilder{x\in S}{x\in A\land (x\in B \lxor x\in C)}\\
&=\setbuilder{x\in S}{(x\in A\land x\in B)\lxor(x\in A\land x\in C)}\\
&=\setbuilder{x\in S}{x\in A\inter B\lxor x\in A\inter C}\\
&=\setbuilder{x\in S}{x\in (A\inter B)+(A\inter C)}\\
&=(A\inter B)+(A\inter C)\\
&=(A\cdot B)+(A\cdot C)
\end{align*}

\subsubsection{Question 14}
First notice that if $A$ and $B$ are disjoint then $m(A\union B) = m(A) + m(B)$. So now we get the three disjoint sets $A\setminus B$, $A\inter B$, and $B\setminus A$, notice that $A =( A\setminus B) \union (A\inter B)$, that $B =(B\setminus A)\union(A\inter B)$, and $A\union B = (A\setminus B)\union (A\inter B)\union (B\setminus A)$. Now we get $m(A) = m(A\setminus B) + m(A\inter B)$, $m(B) = m(B\setminus A) + m(A\inter B)$, and $m(A\union B) = m(A\setminus B)+(A\inter B)+m(B\setminus A)$. We then get \begin{align*}
m(A) + m(B) &= m(A\setminus B)+m(A \inter B)+m(B\setminus A)+m(A\inter B)\\
&= m(A\union B) + m(A\inter B)\\
m(A) + m(B) - m(A \inter B) &= m(A \union B)
\end{align*}

\subsubsection{Question 22}
\question{(a)} To construct a subset of any set we go through each element and choose to include it or not to, this gives us two possibilities per element. For a set of size $n$ then there are $n$ independent choices to be made in constructing a subset, thus $2^n$ subsets.

\question{(b)} There are exactly $\binom nm = \frac{n!}{m!(n-m)!}$ subsets of a set with $n$ elements that have $m$ elements.

\begin{proof}
	Let us start by defining $\binom nm$ as the number of ways to choose a subset with $m$ elements from a set with $n$ elements. Now we must recognize that $k!$ is the number of ways to order a set with $k$ elements. Then we get that $\binom nmm!(n-m)! = n!$ as we may order our set with $n$ elements by choosing the first $m$ elements in our order ($\binom nm$ possible ways), then ordering those elements ($m!$ ways), and finally ordering the rest of the elements ($(m-n!)$ ways). This gives us $\binom nmm!(n-m)!=n!$ and from there we divide and get $\binom nm=\frac{n!}{m!(n-m)!}$.
\end{proof}

\subsection{Section 3}
\subsubsection{Question 7}
Let $g:S\to T$, $h:S\to T$ and $f:T\to U$ be functions such that $f$ is 1-1 and $f\composed g = f\composed h$. Assume for the sake of contradiction that $g\not=h$, then there exists some $s \in S$ such that $g(s) \not= h(s)$. We know that $f\composed g(s) = f\composed h(s)$, thus $f(g(s)) = f(h(s))$ so $g(s) = h(s)$ by $f$ being 1-1. Thus we have a contradiction and we know that $g=h$.

\subsubsection{Question 8}
\question{(a)} Yes, as all integers are either even or odd and none are both even and odd.

\question{(b)} Let us break this into cases:
\begin{itemize}
	\item If $s_1$ and $s_2$ are even, then there exists $k_1\in\mathbb Z$ and $k_2\in\mathbb Z$ such that $2k_1 = s_1$ and $2k_2 = s_1$. Thus $s_1 + s_2 = 2k_1 + 2k_2 = 2(k_1 + k_2)$, thus $f(s_1 + s_2) = 1$. We also find that $f(s_1) \cdot f(s_2) = 1 \cdot 1 = 1$.
	\item If $s_1$ is even and $s_2$ is odd, then there exists $k_1\in\mathbb Z$ and $k_2\in\mathbb Z$ such that $s_1 = 2k_1$ and $s_2 = 2k_2 + 1$. Thus $s_1 + s_2 = 2k_1 + 2k_2 + 1 = 2(k_1 + k_2) + 1$ so $f(s_1 + s_2) = -1$. We also find that $f(s_1) f(s_2) = 1\cdot -1 = -1$.
	\item If $s_1$ is odd and $s_2$ is even we may write that $f(s_1 + s_2) = f(s_2 + s_1)$ and that $f(s_1)f(s_2) = f(s_2)f(s_1)$ because both addition and multiplication are commutative. Now we see that we have reproduced our previous case and thus in this case the equality holds.
	\item If $s_1$ and $s_2$ are odd, then there exists $k_1\in\mathbb Z$ and $k_2\in\mathbb Z$ such that $2k_1 + 1 = s_1$ and $2k_2 + 1 = s_2$, thus $s_1 + s_2 = 2k_1 + 1 + 2k_2 + 1 = 2(k_1 + k_2 + 1)$ so $f(s_1 + s_2) = 1$. We also find that $f(s_1)f(s_2) = -1\cdot -1 = 1$.
\end{itemize}
Thus for all possible integers $s_1$ and $s_2$, we have $f(s_1 + s_2) = f(s_1)f(s_2)$.

This tells us that even integers are closed under addition. that odd integers added together always are even, and finally that an odd added to an even is odd.

\question{(c)} No, as $f(1\cdot 2) = f(2) = 1$ and $f(1)f(2) = -1 \cdot 1 = -1$.

\subsubsection{Question 12}

\question{(a)} No $f$ is not a function as $2/3 = 4/6$ and $f(2/3) = 2^23^3 \not= 2^43^6 = f(4/6)$.

\question{(b)} We may define $f(m/n) = 2^m3^n$ iff $m$ and $n$ are coprime.

\subsubsection{Question 19}

Let $f(x) = x^2+ax+b$, thus $f'(x) = 2x+a$. $f'(x)$ is linear so there exists only one $x\in\mathbb R$ for which $f'(x) = 0$, and thus this $x$ is a global extrema for $f$, so $f$ can not be surjective. Now consider $x_1 = -\frac a2-1$ and $x_2=-\frac a2+1$, thus
\begin{align*}
f(x_1) &= \left(-\frac a2-1\right)^2+a\left(-\frac a2-1\right)+b \\
&= \frac{a^2}4+2\frac a2+1-\frac{a^2}2-a+b\\
&=\frac{a^2}4+1+b\\
f(x_2) &= \left(-\frac a2+1\right)^2+a\left(-\frac a2+1\right)+b \\
&= \frac{a^2}4-2\frac a2+1-\frac{a^2}2+a+b\\
&= \frac{a^2}4+1+b
\end{align*}

so $f$ must be 1-1.

\subsubsection{Question 23}


\def\RB{}
\def\LB{}

\marginnote{Ugly proof:}
First let us show that there exists some bijection from $\mathbb N$ to $\ZNonNegative^2$. Consider the 1 norm on $\ZNonNegative^2$, defined as $\norm{(a,b)}_1 = a + b$. Then we may partition $\ZNonNegative^2$ into subsets $P_n = \setbuilder{x\in\ZNonNegative^2}{\norm x_1 = n}$, for any $n\in \ZNonNegative$. Notice that for $(a,b)\in P_n$ then $a \le n$ and $b \le n$, thus forcing $P_n$ to be finite. Now we can construct a function mapping from $\mathbb N$ to $\ZNonNegative^2$ by giving each element of $P_1$ a number from $1$ to $\card{P_0}$ (inclusive), then the next $\card{P_1}$ will be given to elements of $P_1$ and so on infinitely. Notice that by construction $x\not= y \implies f(x) \not= f(y)$, so we get this being 1-1, additionally for any $(a,b) \in \ZNonNegative^2$, $(a,b)\in P_{a+b}$ and thus receives a number greater than $\sumfrom{n=0}{a+b-1}{\card{P_n}}$ and less than or equal to $\sumfrom{n=0}{a+b}{\card{P_n}}$. This means that we can label each element of $\ZNonNegative$ with a single natural number and thus have a bijection.

Now we can also construct a trivial bijection, $h:\ZNonNegative^2\to S$ as $h(a,b) = 2^a3^b$. Now we may compose the bijections to get a 1-1 correspondence $\mathbb N = S$ onto $T$.

\bigskip

\marginnote{Nice proof:}
First notice that $T \subset S$ so there exists the trivial injective function from $T$ to $S$. Second notice that $f:S\to T$ defined as $f(s)=2^s$ is both well defined as injective. By the  \href{https://en.wikipedia.org/wiki/Schr\%C3\%B6der\%E2\%80\%93Bernstein_theorem}{Schr\"oder-Bernstein theorem} there must be some bijection from $S$ to $T$.

\subsubsection{Question 28}

Let $S$ be a finite set, with $f:S\to S$. Now let $f(x) = f(y)$, for some $x\not=y$, then there remain $\card S - 2$ elements in $S\setminus\set{x,y}$ and $\card S - 1$ elements in $S\setminus\set{f(x)}$. This means that for any definition of $f$ on $S\setminus\set{x,y}$ it can not possibly be onto $S\setminus\set{f(x)}$. We have now shown $f$ not being 1-1 implies $f$ not being onto, by contrapositive $f$ being onto implies $f$ is 1-1.

\subsubsection{Question 29}

Let $S$ be a finite set, with $f:S\to S$ injective. Now as $f$ is 1-1 each $s\in S$ has a unique $f(s)\in S$, so $f(S)$ must have exactly $\card S$ unique elements, thus $f(S) \subset S$ with exactly $\card S$ elements.\footnote{$f(A)$ is defined as $\setbuilder{y\in\range{f}}{\Exists{x\in \domain{f}}{f(x) = y}}$ when $A \subset \domain{f}$ and $A\not\in\domain{f}$.} Because $S$ is finite, this implies $f(S) = S$.

\subsection{Section 4}

\subsubsection{Question 5}

\question{(a)} First identity:
\begin{align*}
f^2g^2&=ffgg\\
&=f(fg)g\\
&=f(gf)f\\
&=(fg)^2
\end{align*}

\question{(b)} Second Identity: Let $i$ be the identity function.
\begin{align*}
\inv f\inv ggf &= i\\
\inv f\inv ggf\inv{(gf)} &= i\inv{(gf)}\\
=\inv f\inv g i&=i\inv{(fg)}\\
=\inv f\inv g&=\inv{(fg)}
\end{align*}

\subsubsection{Question 9}

\question{(a)} 
\begin{align*}
f^2&:x_1\to x_3, x_2 \to x_4, x_3 \to x_1, x_4 \to x_2\\
f^3&:x_1\to x_4, x_2 \to x_1, x_3 \to x_2, x_4 \to x_1\\
f^4&:x_1\to x_1, x_2 \to x_2, x_3 \to x_3, x_4 \to x_4
\end{align*}

\question{(b)}
\begin{align*}
g^2&:x_1\to x_1, x_2 \to x_2, x_3 \to x_3, x_4 \to x_4\\
g^3&:x_1\to x_2, x_2 \to x_1, x_3 \to x_3, x_4 \to x_4
\end{align*}

\question{(c)}
$$fg:x_1\to x_3, x_2\to x_2, x_3\to x_4, x_4\to x_1$$

\question{(d)}
$$gf:x_1\to x_1, x_2 \to x_3, x_3 \to x_4, x_4 \to x_2$$

\question{(e)}
\begin{align*}
(fg)^3&:x_1\to x_1, x_2 \to x_2, x_3 \to x_3, x_4 \to x_4\\
(gf)^3&:x_1\to x_1, x_2 \to x_2, x_3 \to x_3, x_4 \to x_4
\end{align*}

\question{(f)} No, $fg(x_1) \not= gf(x_1)$ as can be seen above, thus $fg\not=gf$.

\subsubsection{Question 10}
Consider the cycle structure of a permutation $f$. It is obvious that $f^k = i$ if $k$ is the greatest common divisor among all the cycle lengths in $f$. Now for any $f\in S_3$, cycles must be of length one, two, or three. Therefore, as $6 = \gcd(1,2,3)$ for any $f\in S_3$, $f^6=i$.

\subsubsection{Question 14}

Let $F$ be the mapping from $S_m\to S_n$ such that $F(f)$ is defined to be the same as $f$ where $f$ is defined, and acts as the identity elsewhere. Now $F$ is trivially 1-1, so let us show that it satisfies $F(fg) = F(f)F(g)$ for all $f,g \in S_m$. To start let us choose $x$ in the domain of $g$, then $F(g)$ takes $x \to g(x)$ and $F(f)$ takes $g(x) \to fg(x)$, which is obviously the same as what $F(fg)$ does. If $x$ is not in the domain of $g$ then $F(g)$ takes $x\to x$ and $F(f)$ takes $x \to x$ as does $F(fg)$, we can thus conclude that $F(fg) = F(f)F(g)$.

\subsubsection{Question 21}

Let $g_j$ swap $x_1$ and $x_{j+1}$. Now when $n = 1$ this is trivially true as we have $f = i$ which satisfies the definition of $f$. Let us now try and do an induction on this statement. Assume that $g_1g_2g_3\cdots g_{n-1} = f$ when $n$ is some specific fixed constant. Then it follows that for $f'\in S_{n+1}$ where $f'$ is defined just as $f$ was, that is $f': x_1\to x_2, x_2 \to x_3, \ldots, x_n \to x_{n+1}, x_{n+1} \to x_1$, then consider $g_1g_2g_3 \ldots g_n = fg_n$ and this will obviously give us $f'$, so by induction we have shown that this may be done for any $n$.


\subsubsection{Question 27}
For every $b$ in the domain of $f$ there must be exactly one $a$ and $c$ such that $f(a) = b$ and $f(b) = c$. As the domain of $f$ is finite then there must be some $n\in\mathbb N$ such that $f^n(b) = b$. It follows then that if there is some $n$ such that $f^n(s) = t$ then there must also be some $k$ such that $f^k(t) = s$. By symmetry we also know that the converse is true. This means that either $O(s) = O(t)$ or the two are disjoint.

\subsubsection{Question 30}
Each orbit must be exactly of size $1$. This is because otherwise all $n$ such that $f^n = i$, would have to be a multiple of a number that is not $1$, and thus could not be any prime number.

\subsubsection{Question 32}
$g\in A(S)$ commutes with $f$ iff $g$ is closed on the set $\set{x_1,x_2}$.
\begin{proof}
	First we will show by cases that any $g$ that is closed on $\set{x_1,x_2}$ commutes with $f$, then we will show that no other set does so.
	
	\begin{itemize}
		\item Let $s,t \in \set{x_1,x_2}$ with $s\not=t$
		\begin{itemize}
			\item If $g(s) = s$, then $fg(s) = g(t) = t$ and $gf(s) = f(s) = t$.
			\item If $g(s) = t$, then $fg(s) = g(t) = s$ and $gf(s) = f(t) = s$.
		\end{itemize}
		\item Let $s \not\in\set{x_1,x_2}$, then $fg(s) = gf(s)$ as $f$ acts as the identity.
	\end{itemize}
	
	Now if $g$ is not closed on $\set{x_1,x_2}$ then lets say without loss of generality that $g(x_1) = s \not\in\set{x_1,x_2}$ it follows that $fg(x_1) = g(x_2)$ and $gf(x_1) = f(s) = s$. Now $g(x_2) \not= s$ as otherwise both $x_1$ and $x_2$ would map to the same element which is not possible.
\end{proof}

\subsection{Section 5}

\subsubsection{Question 1}
For this we use the Euclidean algorithm, rather then do the somewhat tedious math, I will simply employ a program I have written in Python.
\\
\question{(a)} $(116, -84) = 4 = 8 \cdot 116 + 11 \cdot -84$.
\\
\question{(b)} $(85,65)=5=-3\cdot85+4\cdot65$.
\\
\question{(c)} $(72,26)=2=4\cdot72-11\cdot26$.
\\
\question{(d)} $(72,25)=1,8\cdot72-23\cdot25$.

\subsubsection{Question 4}
This shall be nothing but some simple arithmetic, most of these numbers are factorials making them particularly easy to compute.\\
\question{(a)} $36 = 2^23^2$.\\
\question{(b)} $120 = 2^33^15^1$.\\
\question{(c)} $720 = 2^43^25^1$.\\
\question{(d)} $5040 = 2^43^25^17^1$.

\subsubsection{Question 7}

\question{(a)}First, we write $m = k_1(m,n)$ and $n=k_2(m,n)$ for some $k_1,k_2\in\mathbb Z$. It follows $$\frac{mn}{(m,n)} = k_1k_2(m,n) = mk_2 = nk_1$$ so this satisfies $m|v$ and $n|v$.

\newcommand{\factorize}[2]{\prodacross{{#2}\in\mathbb N}{{p_{#2}}^{{#1}_{#2}}}}

\begin{lemma}
	For $n = \factorize ni$ and $m = \factorize mi$, if $c_i = \min(n_i,m_i)$ then $$(n,m) = \factorize{c}{i}$$ where $p_i$ is the $i^{\text{th}}$ prime number. 
\end{lemma}

\begin{proof}
	For convention we will let $p_i$ be the $i^{\text{th}}$ prime unless otherwise stated. We will also adopt the convention that for any natural number $x$, the sequence $x_i$ will be it's prime factorization, that is $\factorize{x}{i} = x$ unless otherwise stated. Furthermore we will also by convention assume that if a sequence of natural numbers $x_i$ has been defined then $x = \factorize{x}{i}$, unless otherwise stated. As a last note, we will define $\mathbb N = \set{0,1,2,\ldots}$ and $2 = p_0$.
	
	Let $n$ and $m$ be natural numbers, and then let $c_i = \min{n_i,m_i}$ for all $i\in\mathbb N$. We would like to show $c=(n,m)$. First it is trivial that $c>0$.
	
	Second we must show $c|n$ and $c|m$. To do this let $k_i = n_i - c_i$, notice that $n_i \ge c_i$ for all $i$, therefore $k_i$ is an integer for all $i$.
	\begin{align*}
	kc &= \factorize ki \factorize ci \\
	&= \prodacross{i\in\mathbb N}{{p_i}^{n_i-c_i}}\factorize ci \\
	&= \prodacross{i\in\mathbb N}{{p_i}^{n_i}} \\
	&= n
	\end{align*}
	The same argument can be made to show that $c|m$.
	
	Lastly we must show that if $d|n$ and $d|m$ then $d|c$, we will do this by contrapositive, so assume $d\nmid c$, therefore there does not exist any $k$ st. $dk = c$. Further there exists no sequence of natural numbers $k_i$ st. $d\factorize ki = c$. We know have
	\begin{align*}
	\factorize ki \factorize di &= \prodacross{i\in\mathbb N}{{p_i}^{d_i+k_i}} \\
	&\not= \factorize ci
	\end{align*}
	for any sequence $k_i$, therefore there must exists some $i\in\mathbb N$ st. $d_i > c_i$. It follows then that either $d_i > n_i$ or $d_i > m_i$.
\end{proof}

Now note that $\min(a,b) + \max(a,b) = a + b$ for any $a,b$. Therefore if we define $v_i = \max(n_i,m_i)$ and $c_i = \min(n_i,m_i)$ we get
\begin{align*}
\frac{mn}{(m,n)} &= \frac{\factorize mi \factorize ni}{\factorize ci} \\
&= \prodacross{i\in\mathbb N}{{p_i}^{m_i+n_i-c_i}} \\
&= \factorize vi \\
&= v
\end{align*}

Now we just need to show that $v$ is the least common multiple. If $r < v$ and $\factorize ri = r$, it follows that is some $i$ for which $r_i < v_i$, therefore either $m$ or $n$ can not possibly divide $r$ as either $m_i > r_i$ or $n_i > r_i$.

We now know that $mn/(m,n)$ is the least common multiple of $m$ and $n$.

\question{(b)} As we have already shown $v = \prodacross{i\in\mathbb N}{{p_n}^{\max(n_i,m_i)}}$.

\subsubsection{Question 13}

\question{(a)} If $p = 4n$ then $p$ is divisible by four an not prime. If $p = 4n + 2 = 2(2n + 1)$ then $p$ is divisible by two and not odd. Therefore either $p = 4n+1$ or $p = 4n+3$.

\question{(b)} If $p = 6n$ then $p$ is divisible by six and not prime. If $p = 6n + 2 = 2(3n + 1)$ then $p$ is divisible by two and not odd. If $p = 6n + 3 = 3(2n + 1)$ then $p$ is divisible by three and is either the number 3 or is not prime. If $p = 6n + 4 = 2(3n + 2)$ then $p$ is divisible by two. Therefore if $p$ is an odd prime that is not $3$, then either $p = 6n+1$ or $p = 6n+5$.

\subsubsection{Question 17}

Let $p$ be the $n^{\text{th}}$ prime. Assume for the sake of contradiction that there is some $a,b\in\mathbb N$ st. $a^2=pb^2$, and let $\factorize ai = a$ and $\factorize bi = b$. It follows that $\factorize{2a}i=p\factorize{2b}i$. As $p$ is the $n^{\text{th}}$ prime then $$p^{2a_n}\prodacross{i\in\mathbb N\setminus\set n}{{p_i}^{2a_i}} = p^{2b_n+1}\prodacross{i\in\mathbb N\setminus\set n}{{p_i}^{2a_i}}$$ so the prime factorizations can not possibly be the same, so we have a contradiction.

\subsection{Section 6}

\subsubsection{Question 1}

\begin{proof}
Base case, we have $\frac161(1+1)(2\cdot 1 + 1) = \frac166 = 1 = 1^2$, when $n = 1$.
Inductive case we get 
\begin{align*}
\frac 16(n-1)((n-1)+1)(2(n-1)+1) + n^2 &=\frac16(n-1)n(2n-1) + n^2 \\
&= \frac16\left(2n^3 - 3n^2+n\right) + n^2 \\
&= \frac16\left(2n^3 + 3n^2+n\right) \\
&= \frac16n(2n^2+3n + 1) \\
&= \frac16n(n+1)(2n+1)
\end{align*}
\end{proof}

\subsubsection{Question 2}

\begin{proof}
	Base case, we have $\frac141^2(1+1)^2 = \frac144 = 1 = 1^3$, when $n = 1$. Inductive case we get 
	\begin{align*}
	\frac14(n-1)^2((n-1)+1)^2 + n^3 &= \frac14n^2(n-1)^2 + n^3 \\
	&= \frac14\left(n^4-2n^3+n^2\right)+n^3\\
	&= \frac14\left(n^4+2n^3+n^2\right)\\
	&= \frac14n^2(n+1)^2
	\end{align*}
\end{proof}

\subsubsection{Question 8}

\begin{proof}
Our base case is trivial when $n=1$. In our inductive case we get
\begin{align*}
\frac{(n-1)}{n} + \frac1{n(n+1)} &= \frac{(n-1)(n+1) + 1}{n(n+1)}\\
&= \frac{n^2}{n(n+1)} \\
&= \frac n{n+1}
\end{align*}
\end{proof}


\subsubsection{Question 14}

Let $n = 0$, then it is trivial that $n^p - n$ is divisible by $p$ for any prime $p$.

Now let $n$ be a fixed non-negative integer, and assume that $n^p-n$ is divisible by $p$ for any prime $p$. By the binomial theorem we have
\begin{align*}
(n+1)^p - (n+1)
&= \sumfrom{i=0}{p}{\binom{p}{i}n^i}-n-1 \\
&= \sumfrom{i=1}{p-1}{\binom{p}{i}n^i}+n^p+1-n-1 \\
&= \sumfrom{i=1}{p-1}{\binom{p}{i}n^i}+\left(n^p-n\right)
\end{align*}
By our assumption we have $n^p-n$ is divisible by $p$. Additionally $\binom pi$ must be divisible by $p$ for all $0 < i < p$ because $\binom pi = \frac{p!}{i!(p-i)!}$ and $p$ is prime.

By induction we then know that $n^p-n$ is divisible by $p$ for any prime $p$.

\subsection{Section 7}
\subsubsection{Question 1}

\question{(a)} $(6-7i)(8+i) = 48 - 56i + 6i + 7 = 55-50i$

\question{(b)} $(\frac 23 + \frac 32i)(\frac23 - \frac 32i) = \frac49+\frac94=\frac{16+81}{36} = \frac{97}{36}$

\question{(c)} $(6-7i)(8-i) = 48-56i-6i-7 = 41-62i$

\subsubsection{Question 2}

In general $z^{-1} = \frac{\bar z}{\abs{z}^2}$

\question{(a)} $z^{-1} = \frac{6}{6^2+8^2} - \frac8{6^2+8^2}i$

\question{(b)} $z^{-1} = \frac{6}{6^2+8^2} + \frac8{6^2+8^2}i$

\question{(c)} $z^{-1} = \frac1{\sqrt2} - \frac1{\sqrt 2}i$

\subsubsection{Question 3}
Using Lemma 1.7.1, the fact that $\overline 1 = 1$, and some group axioms, we get.
\begin{align*}
1 &= (\overline z)^{-1}\overline z \\
\therefore \overline 1 &= \overline{(\overline z)^{-1}\overline z} \\
&= \overline{(\overline z)^{-1}}\cdot\overline{(\overline z)} \\
&= \overline{(\overline z)^{-1}} \cdot z \\
\therefore z^{-1} &= \overline{(\overline z)^{-1}} \\
\therefore \overline{z^{-1}} &= \overline{\left(\overline{(\overline z)^{-1}}\right)} \\
&= (\overline z)^{-1}
\end{align*}

\subsubsection{Question 6}

For any $z \in\mathbb C$, there exists $a,b\in\mathbb R$ such that $z = a+bi$. Now $\overline z = a-bi$ by definition. Therefore $z = \overline z$ iff $b=0$ as $a-bi = a+bi$ iff $b=0$. Finally if $b=0$ then $z = a$ and therefore $z$ is real, if $b\not=0$ then $z=a+bi$ for some non-zero $b\in\mathbb R$ so $z$ has an imaginary part and is not real. Therefore we have shown that $z = \overline z$ iff $z\in\mathbb R$.

Now if $a=0$ the we say that $z$ is purely imaginary as there is no real part to $z$. So if $z$ is purely imaginary then $z = bi$ and $\overline z = -bi = -z$. If we start with $\overline z = -z$ then we get 
\begin{align*}
-(a+bi) &= a-bi \\
-a-bi &= a-bi \\
-a = a\\
a = 0
\end{align*}
so $z$ must be purely imaginary. Putting this all together we get that $-z=\overline z$ iff $z$ is purely imaginary.



%\subsubsection{Question 8}

%By lemma 1.7.3 from the book we know that $\abs{uv} = \abs u\abs v$, therefore, as $zz^{-1} = 1$ we also have $\abs{zz^{-1}} = 1$ so $\abs z\abs{z^{-1}} = 1$ and it follows $\abs {z^{-1}} = 1 / \abs z$.

\subsubsection{Question 11}

\question{(a)} $z = \cos\frac{7\pi}4 + i\sin\frac{7\pi}{4}$

\question{(b)} $z = 4\left(\cos \frac\pi2 + i\sin \frac\pi2\right)$

\question{(c)} $z = 36\left(\cos \frac\pi4 + i\sin \frac\pi4\right)$

\question{(d)} $z = 13\left(\cos \frac{2\pi}3 + i \sin\frac{2\pi}3 \right)$

\subsubsection{Question 13}
\begin{align*}
\left(\frac12 + \frac12\sqrt3i\right)^3 &= \left(\frac12\left(1+\sqrt3i\right)\right)^3 \\
&= \frac18\left(1+\sqrt3i\right)^3 \\
&= \frac18\left( 1 + 3\sqrt3i+ 3\left(\sqrt3i\right)^2 + \left(\sqrt3i\right)^3 \right) \\
&= \frac18\left( 1 + 3\sqrt3i - 9 - 3\sqrt3i \right) \\
&= \frac18\left(-8\right) \\
&= -1
\end{align*}


%\subsubsection{Question 14}

%Get help from Russos?
%By Euler's equation we know that $\cos\theta + i\sin\theta = e^{i\theta}$. We have ${\left(e^{i\theta}\right)}^m = e^{im\theta} = \cos (m\theta) + i\sin (m\theta)$

\subsubsection{Question 20}

Let us adopt the notation that for any $c\in\mathbb C$, $$c = c_a + c_bi$$ where $c_a,c_b\in\mathbb R$.
\begin{align*}
\abs{z+w}^2+\abs{z-w}^2 &= \abs{z_a+z_bi + w_a+w_bi}^2 + \abs{z_a+z_bi - w_a-w_bi}^2 \\
&= (z_a+w_a)^2+(z_b+w_b)^2+(z_a-w_a)^2+(z_b-w_b)^2 \\
&= {z_a}^2+2z_aw_a+{w_a}^2 + {z_b}^2+2z_bw_b+{w_b}^2 +{z_a}^2-2z_aw_a+{w_a}^2 + {z_b}^2-2z_bw_b+{w_b}^2 \\
&= 2\left({z_a}^2+{a_b}^2+{w_a}^2+{w_b}^2\right)\\
&= 2\left( \abs{z}^2 + \abs{w}^2 \right)
\end{align*}

\subsubsection{Question 21}

Our approach here is to partition $A$ into countably many finite sets, this will show that there is a 1-1 and onto correspondence from $A$ to $\mathbb N$ as they are both countably infinite. We define $\abs{a+bi}_1 = \abs a + \abs b$. Now we define the set $Z_k = \setbuilder{z\in A}{\abs{z} = k}$ for $k \in\mathbb N \union\set0$. Now for all $z\in A$, there exists some $k\in\mathbb N \union\set 0$ such that $z\in Z_k$ as for any $z\in A$, $z = a+bi$ for $a,b\in\mathbb Z$ and therefore $\abs{z} = \abs{a} + \abs{b} \in \mathbb N \union \set 0$ so there is some $k\in\mathbb N \union\set 0$ such that $z \in Z_k$. For any $k\in\mathbb N\union\set 0$ we also have $Z_k$ is finite as for all $a+bi \in Z_k$, $\abs a\le k$ and $\abs b \le k$, therefore there are only finitely many possibilities for $a$ and $b$. Now we have $\unionacross{k\in\mathbb N \union\set 0}{Z_k} = A$ with each $Z_k$ finite, so $A$ must be countable.


\subsubsection{Question 22}	

First we will prove that $P(\overline x) = \overline{P(x)}$ for any polynomial $P:\mathbb C \to \mathbb C$ with real coefficients, $\alpha_0, \alpha_1,\ldots,\alpha_n$. Let $z \in \mathbb C$ such that $z = a+bi$ with $a$ and $b$ real. Notice that for any $\alpha\in\mathbb R$, 
\begin{align*}
\alpha\overline z &= \alpha(a - bi) \\
&= \alpha a - \alpha bi \\
&= \overline{\alpha z}
\end{align*}
From lemma 1.7.1 we get $\overline{zw} = \overline z \overline w$, so it follows that $\overline{z^n} = \overline{z}^n$. Finally from lemma 1.7.1 we also get $\overline {z + w} = \overline z + \overline w$ so it follows that $\overline{\sum z_j} = \sum \overline{z_j}$. So if $P(x) = \sumfrom{j=0}{n}{\alpha_jx^j}$ then it follows
\begin{align*}
P(\overline{x}) &= \sumfrom{j=0}{n}{\alpha_j\overline x^j} \\
&= \sumfrom{j=0}{n}{\alpha_j\overline{x^j}} \\
&= \sumfrom{j=0}{n}{\overline{\alpha_jx^j}} \\
&= \overline{\sumfrom{j=0}{n}{\alpha_jx^j}} \\
&= \overline{P(x)}
\end{align*}

Therefore if we have any polynomial $P$ with real coefficients, and $P(x) = 0$ then $P(\overline x) = \overline 0 = 0$.


\section{Chapter 2}
\subsection{Section 1}
\subsubsection{Question 8}
Let us start with when $n=0$, then $(a*b)^n = e = a^n*b^n$.

For $n > 0$ we will do induction, so let us assume that $(a*b)^{n-1} = a^{n-1}*b^{n-1}$, therefore 
\begin{align*}
(a*b)^n &= (a*b)^{n-1}*(a*b) \\
&= (a^{n-1}*b^{n-1}) * (a * b) \\
&= (a^{n-1} * a) * (b^{n-1} * b) \\
&= a^n * b^n
\end{align*}
as we already have the case $n=0$ this induction proves the statement for $n\ge 0$.

Now assume $n<0$, therefore $a^n = \left(a^{-1}\right)^{-n}$ and as $a^{-1} \in G$ and $-n > 0$ then we simply refer to our previous work and conclude that the statement still holds.

\subsubsection{Question 9}

Let $a,b\in G$.
\begin{align*}
e &= (a*b)^2 \\
e &=  a^2 \\
e &= b^2 \\
e &= e * e \\
&= a^2 * b^2\\
a^2 * b^2 &= (a*b)^2\\
a*a*b*b &= a*b*a*b \\
a^{-1}*a*a*b*b*b^{-1} &= a^{-1}*a*b*a*b*b^{-1}\\
a*b &= b*a
\end{align*}

\subsubsection{Question 19}

We simply list off all elements of $S_3$ as $S_3$ is small.

\begin{tabular}{c|c|c}
	$x\in S_3$ & Does $x^2 = e$ & Does $x^3 = e$ \\
	\hline
	$(1,2,3)$ & Yes & Yes \\
	$(1,3,2)$ & Yes & No \\
	$(2,1,3)$ & Yes & No \\
	$(2,3,1)$ & No & Yes \\
	$(3,2,1)$ & Yes & No \\
	$(3,1,2)$ & No & Yes \\
\end{tabular}

\subsubsection{Question 20}
This is all elements $p\in S_4$ such that there does not exist exactly one $x$ such that $p(x) = x$.
\begin{enumerate}
	\item $(1,2,3,4)$
	\item $(1,2,4,3)$
	\item $(2,1,3,4)$
	\item $(2,1,4,3)$
	\item $(1,4,3,2)$
	\item $(3,2,1,4)$
	\item $(3,4,1,2)$
	\item $(1,3,2,4)$
	\item $(4,2,3,1)$
	\item $(4,3,2,1)$
	\item $(2,3,4,1)$
	\item $(2,4,1,3)$
	\item $(3,4,2,1)$
	\item $(3,1,4,2)$
	\item $(4,1,2,3)$
	\item $(4,3,1,2)$
\end{enumerate}
\subsubsection{Question 26}

Let $G$ be a finite group. Assume for the sake of contradiction that there is some $a\in G$ such that for all $n \in\mathbb N$, $a^n\not= e$. As $G$ is finite there then must be some $n_1\not=n_2$ such that $a^{n_1} = a^{n_2}$ by the pigeon hole principle. Let us assume without loss of generality that $n_2 > n_1$, it follows then that $a^{n_1}*a^{-n_1} = a^{n_2}*a^{-n_1}$ and therefore $e = a^{n_2 - n_1}$. This means we have a contradiction and therefore there exists some $n\in\mathbb N$ such that $a^n=e$ for any $a \in G$.

\subsubsection{Question 27}

We already have shown that each element $a\in G$ has some specific $n_a$ such that $a^n = e$. It follows then that if $m = {\prodacross{a\in G}{n_a}}$ then $a^m = e$ for all $a\in G$.
\begin{proof}
	Choose $a \in G$ and let $m = \prodacross{g\in G}{n_g}$. Now $a^m = a^{(m/n_a)(n_a)}$, and for notation let $m_a = \frac m{n_a}$. We now have
	\begin{align*}
	a^m &= a^{n_a\cdot m_a} \\
	&= \left(a^{n_a}\right)^{m_a} \\
	&= e^{m_a} \\
	&= e
	\end{align*}
\end{proof}

\subsubsection{Question 28}

First we know that for any $a \in G$ there exists some $a^{-1} \in G$ such that $a^{-1} a = e$ We will adopt this notation as well as simply saying $ab = a * b$ for $a,b\in G$. Now we have
\begin{align*}
aa^{-1}aa^{-1} &= aea^{-1} \\
&= aa^{-1} \\
\therefore \left(aa^{-1}\right)^{-1}aa^{-1} &= \left(aa^{-1}\right)^{-1}aa^{-1}aa^{-1} \\
\therefore e &= aa^{-1}
\end{align*}

Next we wish to prove some a lemma. If $ab = ac$ then $b = c$
\begin{proof}
	\begin{align*}
	ab = ac &\implies a^{-1}ab = a^{-1}ac \\
	&\implies eb=ec \\
	&\implies b = c
	\end{align*}
\end{proof}

Now with this we can say that for all $a \in G$, there exists exactly one inverse as if $ab = e = ac$, then $b = c$. We also can say an element $a\in G$ is the inverse of exactly one element by the exact same proof.

Finally we get
\begin{align*}
aea^{-1} &= aa^{-1} \\
&= e \\
\therefore a^{-1} &= (ea)^{-1} \\
\therefore a &= ea
\end{align*}
	

\subsection{Section 2}
\subsubsection{Question 1}

Let $a\in G$, therefore there is some $e\in G$ st. $ea = a$ by statement 1. Let $b \in G$ therefore there exists $c\in G$ such that $ca = b$. It follows $be = cae = ca = b$. It has now been shown that there exists $e \in G$ such that for all $x \in G$, $xe = x$.

Now let $a\in G$ then there must exist a $b\in G$ such that $ba = e$. Now we shown that $G$ has the same properties as the set $G$ from Section 1, Question 28, and therefore must be a group.

\subsubsection{Question 2}

Choose $a \in G$, and let us define $f:G\to G$ as $f(x) = ax$ and $g:G\to G$ as $g(x) = xa$. Now the thing to notice is that $f$ is 1-1 as for any $u,v \in G$ we have $f(u) = au$ and $f(v) = av$, so $av = au$ iff $v=u$. The same statement may be made about $g$. As $f$ and $g$ are 1-1 on a finite set $G$ we then also have $f$ and $g$ are onto. This means that for any $a,y\in G$ there is some $x$ such that $ax = f(x) = y$ and for any $a,w\in G$ there is some $u$ such that $ua = g(u) = w$. Therefore we have the conditions from Question 1 and can conclude that $G$ is a group.

\subsubsection{Question 5}

All we need to show that $G$ is abelian is for all $a,b \in G$, $ab = ba$. To start let us choose $a,b \in G$. We know that $a^5b^5 = (ab)^5$ so we get
\begin{align*}
a^5b^5 &= (ab)^5 \\
&= a(ba)^4b \\
\therefore a^4b^4 &= (ba)^4
\end{align*}
and we make a similar argument with $a^3b^3 = (ab)^3$ so
\begin{align*}
a^3b^3 &= (ab)^3 \\
&= a(ba)^2b \\
\therefore a^2b^2 &= (ba)^2
\end{align*}
We combine these to get that
\begin{align*}
a^4b^4 &= (ba)^4\\
 &= \left((ba)^2\right)^2 \\
 &= \left(a^2b^2\right)^2 \\
 &= a^2b^2a^2b^2 \\
 \therefore a^2b^2 &= b^2a^2
\end{align*}
 and finally wrap up with
 \begin{align*}
 a^2b^2&= b^2a^2 \\
 &= (ab)^2 \\
 \therefore a^2b^2 &= (ab)^2 \\
 &= abab \\
 \therefore ab &= ba
 \end{align*}
 
 
\subsection{Section 3}
\subsubsection{Question 4}\label{sec:question-4}
$Z(G)$ is defined as $\setbuilder{z\in G}{x\in G \; \forall\; {zx=xz}}$. First let us choose $a,b\in Z(G)$ therefore for all $x \in G$ we have $ax = xa$ and $bx = xb$. It follows
\begin{align*}
(ab)x &= a(bx) \\
&= a(xb) \\
&= (ax)b \\
&= (xa)b \\
&= x(ab)
\end{align*}
so $ab \in Z(G)$.

Now choose $a\in Z(G)$, therefore for all $x\in G$, $ax = xa$. It follows then that
\begin{align*}
(xa)a^{-1} &= x \\
&= a^{-1}ax \\
&= a^{-1}(xa) \\
xaa^{-1}a^{-1} &= a^{-1}xaa^{-1} \\
\therefore xa^{-1} &= a^{-1}x
\end{align*}
and by definition $a^{-1} \in Z(G)$. Now by lemma 2.3.1 $Z(G)$ is a subgroup of $G$.

\subsubsection{Question 5}
Let $x \in Z(G)$, it follows that for all $a \in G$, $ax = xa$, so therefore for all $a \in G$, $x \in C(a)$, thus $Z(G) \subset \interacross{a\in G}{C(a)}$. Let $x \in \interacross{a\in G}{C(A)}$, then $xa = ax$ for all $a \in G$, and thus $x \in Z(G)$ and therefore $\interacross{a\in G}{C(A)} \subset Z(G)$. By definition of set equality $Z(G) = \interacross{a\in G}{C(A)}$.

\subsubsection{Question 11}
Let $a,b \in H$, therefore $a^{n(a)} = e = b^{n(b)}$. Now it follows that
\begin{align*}
(ab)^{n(a)\cdot n(b)} &= a^{n(a)n(b)}b^{n(a)n(b)} \\
&= e^{n(b)}e^{n(a)} \\
&= e
\end{align*}
thus $ab \in H$.

Let $a \in H$, therefore $a^{n(a)} = e$. It follows that $a^{n(a)-1}a = e$, so $a^{n(a)-1} = a^{-1}$ and as we have already shown $H$ to be closed, then $a^{-1}=a^{n(a)-1}\in H$.

By lemma 2.3.1 it has been demonstrated that $H$ is a subgroup of $G$.

\subsubsection{Question 22}

Let us first show that $AB$ is a group. For any $x,y\in AB$, then let $x_a,y_a \in A$ and $x_b,y_b\in B$ such that $x_ax_b = x$ and $y_ay_b = y$. It follows that $xy = x_ax_by_ay_b = x_ay_ax_by_b \in AB$ as $G$ is abelian. Now for $x \in AB$ then there exists $a\in A$ and $b\in B$ such that $ab = x$. Therefore $a^{-1} \in A$ and $b^{-1}\in B$ as they are both groups, it then follows that $a^{-1}b^{-1} \in AB$ and $a^{-1}b^{-1}ab = e$ by commutativity, so $a^{-1}b^{-1}=(ab)^{-1}$. Thus it has been shown that $AB$ is a group.

Next we will show that $\card{AB}= \frac{\card{A}\card{B}}{\card{A\inter B}}$. 

Let us choose $a \in A$ and $b \in B$ then notice that for all $c \in A\inter B$ then $ac \in A$ and $c^{-1}b \in B$ and therefore $(ac)(c^{-1}b) = ab$. Therefore there are at least as many $(a,b)$ pairs such that $ab$ is equal as there are elements in $A \inter B$. Now assume there is some other $\bar a \in A$ and $\bar b \in B$ such that there is no $\bar c \in A\inter B$ such that $\bar a \bar c = a$ and ${\bar c}^{-1}\bar b = b$. If $\bar a\bar b = ab$ then $a^{-1}\bar a = b\bar b^{-1} \in A\inter B$. We will let $\bar c^{-1} = b\bar b^{-1}$ as we then get $\bar c^{-1}\bar b = b\bar b^{-1}\bar b = b$ we then will also find that $\bar c= \bar a^{-1}a$ and thus $\bar a\bar c = \bar a\bar a^{-1}a = a$. Finally we conclude that for any element $ab \in AB$, there exists exactly $\card{A \inter B}$ pairs $(\bar a,\bar b) \in A\cross B$ such that $ab = \bar a\bar b$.

Finally if $\bar A$ is relatively prime  to $\bar B$ then $A \inter B = \set e$ as for any $a \in A \inter B$ there would be a cyclic set generated by $a$ and that cyclic set's order must divide both $A$ and $B$. Only if the order is one is this possible so the only element may be the identity element.

\subsubsection{Question 24}

First we show $N$ to be a group. Let $n,k \in N$, so if we choose $x \in G$ then there exists $h_1\in H$ such that $n=x^{-1}h_1x$ and $h_2 \in H$ such that $k=x^{-1}h_2x$. It follows that $nk= x^{-1}h_1xx^{-1}h_2x = x^{-1}h_1h_2x$, and therefore is in $N$ as $h_1h_2 \in H$. Now let $n \in N$, and choose $x\in G$ then there exists $h\in H$ such that $n = x^{-1}hx$ then it follows that $n^{-1} = \left(x^{-1}hx\right)^{-1} = x^{-1}h^{-1}x$ and as $h^{-1} \in H$ then $n^{-1} \in N$. This proves $N$ to be a group.

Now let us show that for all $y \in G$, $y^{-1}Ny = N$. Let us choose $n \in N$, and $y \in G$. By $n \in N$ we then choose $x \in G$ and there must exist $h \in H$ such that $n=x^{-1}hx$, then it follows that $y^{-1}ny= y^{-1}x^{-1}hxy$. Now we may choose $z \in G$ and let $x$ be such that $xy = z$ then we find that $y^{-1}x^{-1}hxy = z^{-1}hz$ and therefore $y^{-1}Ny=N$.

\subsubsection{Question 26}

If there exists $h_1, h_2 \in H$ such that $h_1a = h_2b$ then it follows that $ab^{-1} = {h_1}^{-1}h_2 \in H$. Now for any $\bar a \in Ha$ there is some $h \in H$ such that $ha = \bar a$. It then must be so that $hab^{-1}b \in B$ as we know that $ab^{-1} \in H$ and as $hab^{-1}b = ha = \bar a$ then $Ha \subset Hb$. By symmetry we also know that $Hb \subset Ha$, therefore $Ha = Hb$.

We've now shown if $Ha \inter Hb \not= \emptyset$ then $Ha = Hb$ and otherwise obviously $Ha \inter Hb = \emptyset$.

\subsubsection{Question 28}

First let us reference the next question as it will prove that $M = x^{-1}Mx$ and $N = x^{-1}Nx$ for all $x\in G$. Now we will proceed to show that $MN$ is a group.

Let $c \in MN$, therefore there is some $a \in M$ and $b \in N$ such that $ab = c$. Now as $N = x^{-1}Nx$ for all $x \in G$ then there exists some $\bar b \in N$ such that $a^{-1}\bar ba = b$, therefore $ab = aa^{-1}\bar ba = \bar ba$. We now get inverses as $c^{-1} = a^{-1}\bar b^{-1}$ and $a^{-1} \in M$ and $\bar b^{-1} \in N$. Now let $d \in MN$ as well, then there is some $m \in M$ and $n \in N$ such that $d = mn$. Now as $m \in M = x^{-1}Mx$ for all $x \in G$ then there is $\bar m \in M$ such that $b^{-1}\bar mb = m$. It follows that $cd = abmn = abb^{-1}\bar mbn = a\bar mbn\in MN$ as $a,\bar m \in M$ and $b,n \in N$.

Now finally to show that $x^{-1}MNx \subset MN$ for all $x \in G$ we choose $x \in G$ and $d = mn \in MN$ with $m \in M$ and $n \in N$. We then find $x^{-1}mnx = (x^{-1}mx)(x^{-1}nx)$ and of course $x^{-1}mx \in M$ and $x^{-1}nx \in N$.

\subsubsection{Question 29}

Let $m \in M$ and let $x \in G$. We wish to show the existence of $n \in M$ such that $x^{-1}nx = m$. If we let $x^{-1}nx = m$ then we get $n=xmx^{-1}$ and as $x^{-1}\in G$ and $m\in M$ the we get $n \in x^{-1}Mx \subset M$ so $n \in M$. Now we have shown that $m \in x^{-1}Mx$ and thus $x^{-1}Mx = M$.




\subsection{Section 4}
\subsubsection{Question 1}

\begin{enumerate}
	\item $a ~ b$ for $a,b \in \mathbb R$ iff $a - b \in\mathbb Q$.
	\begin{itemize}
		\item \textbf{Reflexivity} $a - a = 0 \in \mathbb Q$, therefore $a \sim a$.
		\item \textbf{Symmetry} If $a\sim b$ then $a-b =q \in\mathbb Q$, therefore $b-a=-q\in\mathbb Q$, so $b\sim a$.
		\item \textbf{Transitivity} If $a\sim b$ and $b \sim c$ then $a-b = q \in\mathbb Q$ and $b - c = p \in\mathbb Q$, therefore $a-c = a-b+b-c = q + p \in\mathbb Q$, so $a \sim c$.
	\end{itemize}
	\item $a \sim b$ for $a,b\in\mathbb C$ iff $\abs a = \abs b$.
	\begin{itemize}
		\item \textbf{Reflexivity} $a \sim a$ is trivial.
		\item \textbf{Symmetry} If $a\sim b$ then $\abs a = \abs b$. By symmetry of equality we have $\abs b = \abs a$ and therefore $a \sim b$.
		\item \textbf{Transitivity} If $a\sim b$ and $b\sim c$ then $\abs a = \abs b = \abs c$ therefore $\abs a = \abs c$ so $a \sim c$.
	\end{itemize}
	\item $a \sim b$ for lines $a,b$ in the plane if $a$ is parallel to $b$.
	\begin{itemize}
		\item \textbf{Reflexivity} Any line $a$ is parallel to itself.
		\item \textbf{Symmetry} If $a$ is parallel to $b$ then $b$ must also be parallel to $a$.
		\item \textbf{Transitivity} If $a$ is parallel to $b$ and $b$ is parallel to $c$ then we would also have $a$ parallel to $c$.
	\end{itemize}
	\item $a \sim b$ for people $a,b$ if $a$'s eye color is the same as $b$'s eye color.
	\begin{itemize}
		\item \textbf{Reflexivity} You have the same eye color as yourself.
		\item \textbf{Symmetry} If $a\sim b$ then $a$'s eye color is $b$'s eye color and therefore $b$'s eye color is $a$'s eye color so $b \sim a$.
		\item \textbf{Transitivity} If $a\sim b$ and $b \sim c$ then $a$'s eye color is $b$'s eye color and $b$'s eye color is $c$'s eye color then it must be that $a$'s eye color is $c$'s eye color so $a \sim c$.
	\end{itemize}
\end{enumerate}

\subsubsection{Question 5}

Let us first show that $a\sim b$ is an equivalence relation.
\begin{itemize}
	\item \textbf{Reflexivity} For any $a \in G$, $a^{-1}a = e \in H$ therefore $a \sim a$.
	\item \textbf{Symmetry} For any $a\sim b \in G$ we get $a^{-1}b \in H$. Now $\left(a^{-1}b\right)^{-1} = b^{-1}a$ must also be a member of $H$, so $b \sim a$.
	\item \textbf{Transitivity} Let $a \sim b$ and $b \sim c$. We get then that $a^{-1}b \in H$ and $b^{-1}c \in H$. It follows then that $\left(a^{-1}b\right)\left(b^{-1}c\right) = a^{-1}c \in H$, so $a \sim c$.
\end{itemize}
Now we show that $\eqclass{a} = aH$.

Let $\alpha \in \eqclass a$, therefore $a \sim \alpha$ so $a^{-1}\alpha \in H$. We then get that $aa^{-1}\alpha = \alpha \in aH$, so $\eqclass a \subset aH$.

Now Let $\alpha \in aH$, therefore there is some $h \in H$ such that $\alpha = ah$. It follows then that $a^{-1}\alpha = a^{-1}ah = h \in H$ so $a \sim \alpha$, and therefore $\alpha\in\eqclass a$ so $\eqclass a \supset aH$.

We now conclude $aH = \eqclass a$.

\subsubsection{Question 18}

Consider the group $U_p$ under multiplication. Notice that for all $0 < n < p$, $n$ is relatively prime to $p$, so all $0 < n < p$ is included. This yields $p-1$, an even number of elements so when we multiply them all together we get $(p-1)!$ and from problem 16 we know that this must be some $x \in U_p$ such that $x^2 \equiv 1 \mod p$.

Now there are only two $n \in U_p$ such that $n^2 \equiv 1 \mod p$, $1, -1$. The proof for this is as follows. Obviously $1^2 = (-1)^2 = 1$. Now assume for some $n \in U_p$, $n^2 \equiv 1 \mod p$ therefore $n^2 -1 = (n+1)(n-1) \equiv 0 \mod p$ so $p$ divides $n+1$ or $p$ divides $n-1$ as $p$ is prime. This leaves only $n \equiv -1 \mod p$ or $n \equiv 1 \mod p$.

Now as only $1^2$ and $(-1)^2$ are $1$ in $U_p$ we get that every element $x \in U_p$ that is not $1$ or $-1$ has a compliment in $U_p$, and therefore $(n-1)! \equiv 1\cdot -1 \mod p$ and this obviously leaves $(n-1) \equiv -1 \mod p$.


\subsubsection{Question 24}

Let $p = 4n+3$ and let $\mathbb Z/p$ be the set of integers mod $p$, and further let us adopt the notation that $\mathbb Z/p^* = \mathbb Z/p \setminus \set 0$. Now assume $x \in \mathbb Z/p$ such that $x^2 \equiv -1 \mod p$. Now we get $x^4 \equiv 1 \mod p$ so $o(x)$ must divide $4$, either 1, 2, or 4. If $o(x) = 1$ then $x=1$ and $x^2\not \equiv -1 \mod p$ unless $p = 2$ which violates $p = 4n+3$. If $o(x) = 2$ then $x^2 = 1 \not\equiv -1$ as $p\not=2$ again. So it must be that $o(x) = 4$. Now $\mathbb Z/p^*$ forms a group with order $p-1$ under multiplication as $p$ is prime, so if $o(x) = 4$ then the cyclic group generated by $x$ would be a subgroup of $Z/p^*$ with order 4. Now $p-1 = 4n+2$ which is not divisible by 4 so we have a contradiction due to Lagrange's theorem.

This does overlook if $x = 0$, however then $x^2 = 0 \not = -1$.

\subsubsection{Question 30}
\begin{itemize}
	\item $b^2 = aba^4$
	\item $b^4 = \left(b^2\right)^2 = aba^4aba^4 = ab^2a^4 = a^2ba^3$
	\item $b^8 = \left(b^4\right)^2 = a^2ba^3a^2ba^3 = a^2b^2a^3 = a^3ba^2$
	\item $b^{16} = \left(b^8\right)^2 = a^3ba^2a^3ba^2 = a^3b^2a^2 = a^4ba$
	\item $b^{32} = \left(b^{16}\right)^2 = a^4baa^4ba = a^4b^2a = ebe = b$
\end{itemize}

So we know $b^{32} = b$, then it follows that $b^{32}b^{-1} = bb^{-1} = e$ so $b^{31} = e$. This means $o(b)$ must be a divisor of $31$, however as $31$ is prime and we know $o(b)\not= 1$ as $b \not= e$ then $o(b) = 31$.
\subsubsection{Question 35}

For any permutation we may find it's order by looking at it's cycle structure and taking the least common multiple of all the cycles in the permutation. Now if a permutation has a prime order then it's cycles may either be length 1 or length $p$, where $p$ is that specific prime. Not all our cycles can be of length $p$ as $\card S$ is not a multiple of $p$. We conclude there must be some cycle of length 1 and thus some element maps to itself.

\subsubsection{Question 37}
Let $G$ be a cyclic group of order $n$ with $g$ as a primitive element. Choose $m$ such that $m$ is a divisor of $n$ and choose $k \le m$ such that $\gcd(k,m) = 1$. It follows that $\left(g^{k\frac nm}\right)^m = g^{k\frac nmm} = g^{kn} = e$ so $o\left(g^{k\frac nm}\right) | m$. Let us assume for the sake of contradiction now that $o\left(g^{k\frac nm}\right) \not= m$, therefore $o\left(g^{k\frac nm}\right) < m$. Let then $\bar m < m = o\left(g^{k\frac nm}\right)$, therefore we get $\left(g^{k\frac nm}\right)^{\bar m} = g^{k\frac nm \bar m} = e$ so we get that $k\frac nm\bar m \equiv 0 \mod n$ and therefore there is some $\bar k$ such that $k\frac nm\bar m = \bar kn$. Dividing by $n$ e get $\frac {k\bar m}m = \bar kn$. We know that $k$ is relatively prime to $m$ so $\frac{\bar m}m$ is an integer which yields a contradiction as $\bar m < m$ and therefore $\frac{\bar m}{m} < 1$. This means that for all $m$ divisible by $n$, and any $k \le m$ such that $\gcd(k,m) = 1$, we get $o\left(g^{k\frac nm}\right) = m$.

Next we show that for all $a$ with $0 < a \le n$ there is some $m | n$ and $k$ with $\gcd(m,k) = 1$ such that $a = \frac nmk$. For notational sake let for all $x \in\mathbb N$, $x = \factorize{x}{i}$, where $p_i$ is the $i^{\text{th}}$. This means that we wish to show that $$\factorize ai = \frac{\factorize ni}{\factorize mi} \factorize ki$$, we may simplify this and show $$\factorize{a}{i} = \prodacross{i\in\mathbb N}{{p_i}^{n_i-m_i+k_i}}$$ and due to properties of primes we need only show that for all $i\in\mathbb N$, $a_i = n_i-m_i+k_i$. Now we have the restriction that $m|n$ which simply means that for all $i\in\mathbb N$, $m_i \le n_i$. We also have the restriction that $\gcd(m,k) = 1$ this can be taken to mean there are no prime divisors shared between $m$ and $k$ so $k_i\not=0\implies m_i = 0$ and $m_i\not=0\implies k_i=0$ for all $i\in\mathbb N$. With these restrictions we can construct $m$ and $k$. For all $i\in\mathbb N$ we follow these rules:
\begin{itemize}
	\item If $0\le a_i \le n_i$, then let $m_i = n_i - a_i$ and $k_i = 0$, therefore we get $n_i-m_i+k_i = n_i - (n_i-a_i) + 0 = a_i$.
	\item If $n_i < a_i$ then let $k_i = a_i - n_i$ and $m_i = 0$ therefore we get $n_i-m_i+k_i = n_i-0+(a_i-n_i) = a_i$.
\end{itemize}
Notice also that $k \ge m$ as otherwise $a = \frac nmk = n \frac km < n$ and we know $a \le n$.


Finally this means that for any $\alpha \in G$ we know that there exists $0 < a \le n$ such that $g^a = \alpha$ and therefore there is some $m |n$ such that there is a $k\le m$ where $\gcd(k,m) = 1$ and $\frac mnk = a$ so $\alpha = g^{\frac mnk}$ and therefore as we have already shown $o(\alpha) = m$. This means that for all $m | n$ there are exactly $\varphi(m)$ elements $\alpha \in G$ such that $o(\alpha) = m$.

\subsubsection{Question 38}

Let there be a cyclic group $G$ of order $n$. One must exist for all $n \in\mathbb N$ as the integers mod $n$ under addition are a cyclic group of order $n$ with 1 as their primitive root. For each divisor of $m$ of $n$ there are $\varphi(m)$ elements $\alpha \in G$ such that $o(\alpha) = m$. There can be no elements $\alpha\in G$ such that $o(\alpha)$ does not divide $n$ by Lagrange's theorem so we know that $n = \sumacross{m|n}{\varphi(m)}$.

\subsubsection{Question 42}

Let $p = 4n+1$.

\begin{align*}
\frac{p-1}2! = \frac{4n}2! &= (2n)! \\
&= 1 \cdot 2 \cdots 2n \\
&= (1\cdot (n+1))\cdot (2\cdot(n+2)) \cdots (n\cdot(n+n))\\
&= (-1\cdot -(n+1)) \cdot (-2\cdot -(n+2)) \cdots (-n\cdot-(n+n))\\
&\equiv ((p-1)\cdot(p-(n+1)) \cdot ((p-2) \cdot (p-(n+2)) \cdots ((p-n) \cdot (p-(n+n))) \mod p \\
&\equiv (p - 1) \cdot (p - 2) \cdots (p - n) \cdot (p-(n+1)) \cdot (p - (n+2)) \cdots (p-(n+n)) \mod p \\
&\equiv (p-1) \cdot (p-2) \cdots (p-2n) \mod p \\
&\equiv (4n) \cdot (4n-1) \cdots (2n+1) \mod (p = 4n+1) \\
&\equiv \frac{(4n)!}{(2n)!} \mod p \\
\therefore \left(\frac{p-1}2!\right)^2 &\equiv (2n)! \cdot \frac{(4n)!}{(2n)!} \mod p \\
&\equiv (4n)! \mod p \\
&\equiv (p-1)! \mod p \\
\end{align*}

Now by wilson's theorem we may state that if $p$ is prime with $p = 4n+1$ then $\frac{p-1}2! \equiv -1 \mod p$

\subsubsection{Question 43}

Let $G$ be an abelian group with order $n$ and elements $a_1,a_2,\ldots,a_n$ and let $x = a_1a_2\cdots a_n$.

\question{(a)} Suppose $G$ has exactly one element $b \not = e$ such that $b^2 = e$. Then it follows that for all elements $g \in G$ with $g \not = b$ and $g \not = e$ we have $g\not=g^{-1}$. This means that every element except $b$ and $e$ has it's inverse in the product that gives us $x$ so this reduces to $x = bee^{\frac{n-1}2} = b$.

\question{(b)} Consider the $B = \setbuilder{b \in G}{b^2 = e} \subset G$. $B$ is a subgroup of $G$ as for any $a,b \in B$, $(ab)^2 = a^2b^2 = e$ and for any $a \in B$, $a = a^{-1}$. Now for our problem we suppose that $\card{B} > 2$ and as this is the case we may take some element $b_1 \in B$ and we get a cyclic subgroup $B_1 = \set{{b_1}^0,{b_1}^1} \subset B$. Now if $B_1 \not = B$ then there exists some $b_2 \in B\setminus B_1$ and this generates $B_2 = \set{{b_2}^0,{b_2}^1}$ and it follows that $B_1B_2 = \setbuilder{{b_1}^{i_1}{b_2}^{i_2}}{\All{k\in\set{1,2}}{i_k \in \set{0,1}}}$. Now by some sort of induction we will find that $B = B_1B_2\cdots B_k$ where $B_i = \set{e, b_i}$ and $b_i \not\in B_1B_2\cdots B_{i-1}$ This means that for all $b \in B$, $b = \prodfrom{r = 1}{k}{{b_r}^{i_r}}$ with $i_r \in \set{0,1}$ for all $r$ and for all $(i_1,i_2,\ldots, i_k)$ with $i_r \in \set{0,1}$, we get a unique $\prodfrom{r=1}{k}{{b_r}^{i_r}} \in B$. That is to say that $f:\set{0,1}^k \to B$ defined as $f(r) = \prodfrom{r=1}{k}{{b_r}^{i_r}}$ is 1-1 and onto. This means that if we take the product of all these elements we get
\begin{align*}
\prodacross{b \in B}b &= \prodacross{i \in \set{0,1}^k}{\left[\prodfrom{r=1}{k}{{b_r}^{i_r}}\right]} \\
&= \prodfrom{r = 1}{k}{\left[\prodacross{i \in \set{0,1}^k}{{b_r}^{i_r}}\right]}\\
&= \prodfrom{r = 1}{k}{\left[\left({b_r}^0\right)^{2^{k-1}}\left({b_r}^1\right)^{2^{k-1}}\right]} \\
&= \prodfrom{r = 1}{k}{{b_r}^{2^{k-1}}}
\end{align*}

If $k=1$ we get the result from part (a) where $B = \set{b_1,e}$ and our product simplifies to just $b_1$. If $k > 1$ then we get ${b_r}^{2^{k-1}} = \left({b_r}^2\right)^{2^{k-2}} = e^{2^{k-2}}$. Now for any $G$ we would have $x = \prodacross{b\in B}{b}$ where $B = \setbuilder{g\in G}{g^2 = e}$ as all other elements will be paired up with inverses and cancel out. So if there is more than one element in $B$ then we get $x = \prodacross{b\in B}{b} = e$.

\question{(c)} If $n$ is odd we get that there can be no subgroup of $G$ with order two, so if there is any element $b\in G$ such that $b = b^{-1}$ we would get the cyclic group $B$ formed by $b$ would be of order two as $b^2 = e$. This means that if $n$ is odd all elements in $G$ have an inverse that is not themselves, unless that element is $e$ itself so we end with $x = e$ after everything has canceled out.

\subsection{Section 5}
\subsubsection{Question 3}

Let $L_a: G\to G$ be defined as $L_a(x) = xa^{-1}$.
\question{(a)} Show that $L_a \in A(G)$, this is equivalent to showing that $L_a$ is 1-1 as we already know $L_a:G\to G$.

Let $b,c \in G$ and if $L_a(b)=L_a(b)$ then $ba^{-1} = ca^{-1}$ and it follows by cancellation that $b=c$, therefore $L_a$ is 1-1 and $L_a\in A(G)$

\question{(b)} Show that $L_aL_b = L_aL_b$

Let $a,b,x\in G$. It follows that
\begin{align*}
L_aL_b(x) &= L_a(L_b(x)) \\
&= L_a(xb^{-1})\\
&=xb^{-1}a^{-1}\\
&=x(ab)^{-1}\\
&=L_{ab}(x)
\end{align*}
therefore $L_aL_b = L_{ab}$

\question{(c)} Let $\psi:G\to A(G)$ be defined as $\psi(a) = L_a$. Show that $\psi$ is a monomorphism.

\begin{proof}
	We know $\psi$ is a homomorphism as if $a,b\in G$ then $\psi(a)\psi(b) = L_aL_b = L_{ab} = \psi(ab)$.
	
	To show $\psi$ is 1-1 let $a,b\in G$ such that $\psi(a) = \psi(b)$, we then get $L_a=L_b$. If we choose $x\in G$ then
	\begin{align*}
	L_a(x) &= L_b(x) \\
	\therefore xa^{-1}&=xb^{-1} \\
	\therefore a^{-1}&=b^{-1} \\
	\therefore a&=b
	\end{align*}
	so we may conclude that $\psi$ is 1-1 and thus a monomorphism.
\end{proof}

\subsubsection{Question 17}

We already know that the intersection of subgroups is a group so we simply need show that $M\inter N$ is normal if $M$ and $N$ are normal subgroups of $G$. This means we need to show that for all $x\in M\inter N$ and for all $g \in G$, $g^{-1}xg \in M\inter N$. Let us start by choosing an arbitrary $x \in M\inter N$ and $g\in G$ We find that $g^{-1}xg \in M$ by the fact that $M$ is normal and $x \in M$ and the same argument goes for $g^{-1}xg \in N$ so $g^{-1}xg \in M\inter N$ and therefore $M\inter N\normsubgroup G$.

\subsubsection{Question 18}

Let $H$ be a subgroup of $G$ and $N = \interacross{a\in G}{a^{-1}Ha}$. We will show that $N\normsubgroup G$.

Let $n\in N$, by definition on $N$ we have $\All{a\in G}{\Exists{h\in H}{\left(a^{-1}ha = n\right)}}$. Now if we choose $g\in G$ and we find that if we choose $a \in H\subset G$ then there exists $h \in H$ such that $n = a^{-1}ha \in H$ and therefore $g^{-1}ng \in N$. We now can conclude that for all $g \in G$, $g^{-1}Ng \subset N$ and therefore $N\normsubgroup G$.

\subsubsection{Question 19}

\question{(a)} First let us show $H\subset N(H)$. Let $h \in H$ and $\ell \in H$ then $h^{-1}\ell h \in H$ and therefore $h^{-1}Hh\subset H$. If $\ell \in h^{-1}Hh$ then there is $k \in H$ such that $\ell=h^{-1}Kh \in H$ and therefore $h^{-1}Hh \subset H$. It follows that $h^{-1}Hh = H$ and thus $H \subset N(H)$.

Now let us show that $N(H)$ is a subgroup of $G$. If $n \in N(H)$ then $n \in a^{-1}Ha$ for all $a \in G$. This implies there is some $h \in H$ such that $n = a^{-1}ha$ and therefore $n^{-1} = \left(a^{-1}ha\right)^{-1} = a^{-1}h^{-1}a$. We then know that for all $n \in N(H)$, $n^{-1}\in N(H)$. If $n,m\in N(H)$ then for any $a\in G$ there is some $h,k\in H$ such that $n = a^{-1}ha$ and $m = a^{-1}ka$ and therefore $nm = a^{-1}haa^{-1}ka = a^{-1}hka$ and as $hk\in H$ then $nm \in N(H)$ so $N(H)$ must be a subgroup of $G$.

\question{(b)} Let us choose an arbitrary $x \in N(H)$, therefore we get $x^{-1}Hx = H$. It follows by definition that $H\normsubgroup N$.

\question{(c)} Let $K$ be a subgroup of $G$ such that $H\normsubgroup K$. If we choose $k \in K$ then we get $k^{-1}Hk = H$ and thus $k \in N(H)$. It follows that $K \subset N(H)$.

\subsubsection{Question 24}

For notation we will adopt the practice that if $x \in A\cross B$ then $x = \pair{x_1}{x_2}$ with $x_1 \in A$ and $x_2 \in B$.

\question{(a)}
Let $a,b \in G = G_1\cross G_2$, therefore $ab = \pair{a_1}{a_2}\pair{b_1}{b_2} = \pair{a_1b_1}{a_2b_2}$ and as $a_1b_1 \in G_1$ and $a_2b_2 \in G_2$ we get $ab \in G$, so $G$ has closure.

Let $a,b,c\in G$, we find
\begin{align*}
(ab)c &= \left(\pair{a_1}{a_2}\pair{b_1}{b_2}\right)\pair{c_1}{c_2} \\
&= \pair{a_1b_1}{a_2b_2}\pair{c_1}{c_2} \\
&= \pair{a_1b_1c_1}{a_2b_2c_2} \\
a(bc) &= \pair{a_1}{a_2}\left(\pair{b_1}{b_2}\pair{c_1}{c_2}\right) \\
&= \pair{a_1}{a_2}\pair{b_1c_1}{b_2c_2} \\
&= \pair {a_1b_1c_1}{a_2b_2c_2} \\
\therefore (ab)c &= a(bc)
\end{align*}
so we have associativity.

Consider $\pair{e_1}{e_2}$ where $e_1$ is the identity of $G_1$ and $e_2$ is the identity of $G_2$. Choose $a\in G$ and we find
\begin{align*}
a\pair{e_1}{e_2} &= \pair{a_1}{a_2}\pair{e_1}{e_2} \\
&= \pair{a_1e_1}{a_2e_2} \\
&= \pair{a_1}{a_2} \\
&= a \\
\pair{e_1}{e_2}a &= \pair{e_1}{e_2}\pair{a_1}{a_2} \\
&= \pair{e_1a_1}{e_2a_2} \\
&= \pair{a_1}{a_2} \\
&= a \\
\end{align*}
therefore we have an identity $e = \pair{e_1}{e_2}$.

Let $a \in G$. We find that $a\pair{{a_1}^{-1}}{{a_2}^{-1}} = \pair{a_1{a_1}^{-1}}{a_2{a_2}^{-1}} = \pair{e_1}{e_2} = e$ and $\pair{{a_1}^{-1}}{{a_2}^{-1}}a = \pair{{a_1}^{-1}a_1}{{a_2}^{-1}a_2} = \pair{e_1}{e_2} = e$ and therefore we have inverses with $a^{-1} = \pair{{a_1}^{-1}}{{a_2}^{-1}}$.

\question{(b)} Let $x,y\in G_1$. We get $\varphi_1(x)\varphi_1(y) = \pair x{e_2}\pair y{e_2} = \pair{xy}{e_2} = \varphi_1(xy)$, so $\varphi_1$ is a homomorphism.

Let $x,y \in G_1$ such that $\varphi_1(x) = \varphi_1(y)$. It follows that $\pair{x}{e_2} = \pair{y}{e_2}$ and thus $x=y$, so $\varphi_1$ is a monomorphism.

\question{(c)} Let us define $\varphi_2:G_2\to G$ as $\varphi_2(x) = \pair{e_1}x$. Now we can make the exact same argument to show $\varphi_2$ is a monomorphism as we did for $\varphi_1$ and thus $\varphi_2$ is a monomorphism by symmetry.

\question{(d)} It is trivial that $\varphi_1(G_1)\varphi_2(G_2) \subset G$ as $G$ is the co-domain of both $\varphi_1$ and $\varphi_2$ and additionally $G$ is closed.

Choose $g \in G$, therefore $g = \pair{g_1}{g_2}$ with $g_1 \in G_1$ and $g_2 \in G_2$. It follows that $\pair{g_1}{g_2} = \pair{g_1e_1}{e_2g_2} = \pair{g_1}{e_2}\pair{e_1}{g_2} = \varphi_1(g_1)\varphi_2(g_2)$ and thus $G \subset \varphi_1(G_1)\varphi_2(G_2)$ so $G=\varphi_1(G_1)\varphi_2(G_2)$.

Now consider $\pair{g_1}{g_2}\in\varphi_1(G_1)\inter\varphi_2(G_2)$. Now for all $x \in G_1$, $\varphi_1(x) = \pair{x}{e_2}$ and thus $g_2 = e_2$. For all $x \in G_2$, $\varphi_2(x) = \pair{e_1}{x}$ and thus $g_1 = e_1$ so we have $\pair{g_1}{g_2} = e$ for all $\pair{g_1}{g_2} \in \varphi_1(G_1)\inter\varphi_2(G_2)$. It follows that $\set e = \varphi_1(G_1)\inter\varphi_2(G_2)$.

\question{(e)} Let us define $f: G_1\cross G2 \to G_2\cross G_1$ as $f(g) = f(\pair{g_1}{g_2}) = \pair{g_2}{g_1}$. We know $G_1\cross G_2$ is a group and by symmetry $G_2\cross G_1$ must also be a group. Now if we choose $a,b \in G_1\cross G_2$ then we get
\begin{align*}
f(a)f(b) &= f(\pair{a_1}{a_2})f(\pair{b_1}{b_2}) \\
&= \pair{a_2}{a_1}\pair{b_2}{b_1} \\
&= \pair{a_2b_2}{a_1b_1} \\
&= f(\pair{a_1b_1}{a_2b_2}) \\
&= f(ab)
\end{align*}
so $f$ is a homomorphism, however we may go a step further as if $f(a) = f(b)$ then we get $f(\pair{a_1}{a_2}) = f(\pair{b_1}{b_2})$. By definition of $f$ we get $\pair{a_2}{a_1} = \pair{b_2}{b_1}$ and therefore we get $a_2 = b_2$ and $a_1 = b_1$ so it must be that $a = b$ yielding $f$ to be a monomorphism. Now we cay say $G_1\cross G_2 \simeq G_2\cross G_1$.

\subsubsection{Question 26}
\question{(a)} Let $a,b,x \in G$ then
\begin{align*}
\sigma_a\sigma_b(x) &= \sigma_a(\sigma_b(x)) \\
&= \sigma_a\left(bxb^{-1}\right) \\
&= abxb^{-1}a^{-1} \\
&= (ab)x(ba)^{-1} \\
&= \sigma_{ab}(x)
\end{align*} so it follows that $\psi(a)\psi(b) = \sigma_a\sigma_b = \sigma_{ab} = \psi(ab)$ so $\psi$ is a homomorphism.

Now let us show that $\ker(\psi) = Z(G)$.

First  let $x \in Z(G)$, and $a,y \in G$. We then get 
\begin{align*}
\sigma_y\sigma_x(a) &= yxax^{-1}y^{-1}\\
& = yaxx^{-1}y^{-1}\\
& = yay^{-1}\\
& = \sigma_y(a)
\end{align*}
and therefore $\psi(y)\psi(x) = \psi(y)$, thus $x\in\ker(\psi)$ and $Z(G) \subset \ker(\psi)$.

Now let $a \in \ker(\psi)$ It follows then that $\psi(a)\psi(e) = \psi(e)$. Therefore if we choose $x \in G$ we get $\sigma_a\sigma_e(x) = aexe^{-1}a^{-1} = exe^{-1} = \psi(e)$. If we simplify somewhat we get $axa^{-1} = x$ and therefore $ax=xa$. This implies $x \in Z(G)$ and thus $\ker(\psi) \subset Z(G)$.

We are now finished with $Z(G) = \ker(\psi)$.
\subsubsection{Question 29}
\question{(a)}
Let $m \in M$ and we must show that for all $a \in G$, $a^{-1}ma\in M$.

We have $T_a(x) = a^{-1}xa$ is an automorphism as 
\begin{align*}
T_a(x)T_a(y) &= a^{-1}xaa^{-1}ya \\
&= a^{-1}xya 
&= T_a(xy)
\end{align*}
and if $T_a(x) = T_a(y)$ then $a^{-1}xa = a^{-1}ya$ and thus $x=y$ and for all $x \in G$ $T_a(axa^{-1})=a^{-1}axa^{-1}a=x$.

It follows that if $m \in M$ then $a^{-1}ma = T_a(m) \in M$ by definition of $M$ and thus $M\normsubgroup G$.

\question{(b)} If $a \in MN$ then $mn = a$ for some $m \in M$ and $n \in N$. Therefore for any automorphism $\varphi$ we have $\varphi(a) = \varphi(mn) = \varphi(m)\varphi(n)$ and we know $\varphi(m) \in M$ and $\varphi(n)\in N$ so $\varphi(a)=\varphi(m)\varphi(n) \in MN$. Thus $MN$ is a characteristic subgroup of $G$.

\question{(c)} Let $A$ be a group. We already have shown then that $A\cross A = A^2$ is a group when $\pair ab\pair xy = \pair{ax}{by}$. We also get that $A\cross\set e$ is a subgroup of $A^2$ as for $\pair ae\in A\cross\set e$, and $\pair be \in A\cross\set e$, we have $\pair ae\pair be = \pair{ab}e\in A\cross\set e$ and $\pair{a^{-1}}e\pair ae = \pair ee$. Additionally $A$ is normal as for any $a =\pair xy\in G$ and $b = \pair ze\in A\cross\set e$ we have $a^{-1}ba = \pair{x^{-1}}{y^{-1}}\pair ze\pair xy = \pair{x^{-1}zx}{y^{-1}ey} = \pair{x^{-1}zx}{e} \in A\cross\set e$. However $A\cross\set e$ is not a characteristic subgroup of $A^2$ as for $\phi:A^2\to A^2$ defined as $\phi(\pair ab) = \pair ba$ we get $\phi(\pair ae) = \pair ea$ and if $A\not=\set e$ then there exists some $a$ for which $\pair ea \not \in A \cross \set e$.

\subsubsection{Question 38}
\question{(a)}
Let $a,b \in G $ and $q \in S$ therefore there is some $x \in G$ such that $q = Hx$
\begin{align*}
T_aT_b(q) &= T_aT_b(Hx) \\
&= T_a(T_b(Hx))\\
&= T_a(Hxb) \\
&= Hxba \\
&= T_{ba}(Hx) \\
&= T_{ba}(q)
\end{align*}
So $T_aT_b = T_{ab}$, so if we define $\psi: G\to A(S)$ as $\psi(x) = T_x$ then we get a homomorphism.

\question{(b)} If $x \in \ker(\psi)$ we get $\psi(b) = \psi(ba)$ for all $b\in G$. It follows that $T_b(Hx)=T_{ba}(Hx)$ for all $x\in G$. We then get $Hxb = Hxba$ and therefore $T_{a}(Hxb) = Hxb$ for all $x,b$ so $T_a$ is the identity if $a \in \ker(\psi)$. We also get the converse as if $T_a(Hx) = Hx = T_e(Hx)$ then $\psi(a) = \psi(e)$.

Further then that we may state that if $x \in \ker(\psi)$ then we get $T_x(Ha) = Ha$ as $T_x$ for all $a\in G$ as is the identity. It follows that $Hax = Ha$ and therefore $Haxa^{-1} = H$. Thus for any $h \in H$ there is $\bar h \in H$ such that $haxa^{-1} = \bar h$ and thus $axa^{-1}=h^{-1}\bar h \in H$. This also implies that $x \in a^{-1}Ha$ for all $a \in G$ and thus $\ker(\psi) \subset \interacross{a\in G}{a^{-1}Ha}$. Now if $x \in \interacross{a \in G}{a^{-1}Ha}$ then $x \in a^{-1}Ha$ for all $a \in G$ and therefore $axa^{-1} \in H$. We now can say that $Haxa^{-1} = H$ and thus $Hax = Ha$ so $T_x(Ha) = Ha$ for any $ a\in G$. This means that $T_x$ is the identity and thus $x \in \ker(\psi)$ and we can finish by stating $\ker(\psi) = \interacross{a \in G}{a^{-1}Ha}$

\question{(c)} We know $\ker(\psi) = \interacross{a \in G}{a^{-1}Ha} \subset e^{-1}He = H$ so $\ker(\psi)$ is obviously a subset of $H$. We also know $\ker(\psi) \normsubgroup G$ as it is the kernel of a homomorphism. Now if there is a $K$ such that $K \normsubgroup G$ and $K\subset H$ then for all $k \in K$ and $g \in G$ we have $gkg^{-1} \in K$. It then follows that $gkg^{-1} \in H$, so $k \in g^{-1}Hg$ so $k \in\ker(\psi)$.

\subsubsection{Question 42}

As we did in the last problem let us define $S =\setbuilder{Ha}{a \in G}$ and $T_a:S\to S$ as $T_a(Hx) = Hxa$. We may also define $\psi:G\to A(S)$ as $\psi(a) = T_a$.

Now first we need to show that $\card S = 4$. We know this as we may define $a\sim b$ if $Ha = Hb$. This is nearly trivially an equivalence relation so I won't bother showing that however we will find that $\eqclass{a} = \setbuilder{ha}{h\in G}$. The proof is as follows: Consider $x \in \eqclass{a}$, it then follows that $Hx = Ha$ so if we choose $h \in H$ then there is some $\bar h \in H$ such that $ha = \bar hx$ and therefore $a = (h^{-1}\bar h)x \in \setbuilder{ha}{h \in H}$. Consider $x \in \setbuilder{ha}{h \in H}$, therefore $x = ha$ for some $h\in H$ and thus $Hx = Hha = Ha$ so $x \sim a$. This means that for each $x \in G$, $\card{\eqclass{x}} = \card H = 9$, so there must be $4$ equivalence classes and as each equivalence class corresponds to a co-set of $H$ we get 4 co-sets of $H$.

Now as $\card{S} = 4$ then it follows that $\card{A(S)} = 4!$. Now we need a lemma

\begin{lemma}
	Let $f:A\to B$ be a homomorphism. If $\ker(f) = (e)$ then $f$ is a monomorphism.
\end{lemma}
\begin{proof}
	Suppose $f$ is not 1-1, therefore there is some $a\not=b \in A$ such that $f(a) = f(b)$. It follows then that $f(ab^{-1}) = f(a)f(b^{-1}) = f(b)f(b^{-1}) = f(bb^{-1}) = f(e)$ so $ab^{-1}\not=e$ is in $\ker(f)$. Therefore by contrapositive we get if $\ker(f) = (e)$ then $f$ is 1-1. 
\end{proof}

We know our function $\psi$ is not a monomorphism as $\card G > \card{A(S)}$. This means that $\ker(\psi) \not= (e)$. Now the last thing we need to show is $\ker(\psi) \subset H$. We can show this by considering $x \in G \setminus H$ and we find that in order for $x\in \ker(\psi)$ we would expect $\psi(x)\psi(y) = \psi(y)$ for all $y\in G$ which therefore means $T_yT_x(Ha) = T_y(Ha)$ for all $a \in G$. Consider $a = e$ and we get $Hexe = Hee$ so $Hx = H$ and this would mean $x \in H$ contradicting our previous statement. So as we know $\ker(\psi) \subset H$ and $\ker(\psi) \not= (e)$ and we already know that the kernel of a homomorphism is a normal subgroup, so as it is contained in $H$ which is of order 9, then this subgroup must be of order 3 or 9. This concludes our proof that either $H\normsubgroup G$ or there is $N \subset H$ with $\card N = 3$ such that $N \normsubgroup G$.

\subsection{Section 6}
\subsubsection{Question 11}
Suppose $G$ is a group and let $Z(G) = Z$. It is trivial to show that $Z \normsubgroup G$ so suppose further that $G/Z$ is cyclic. As $G/Z$ is cyclic it has some primitive element, $\eqclass g$. Now for any $x,y \in G$ we have $x \in Zg^n$ and $y \in Zg^m$ for some $n,m\in\mathbb N$ as all elements are in some coset of $Z$. We also have then that for some $z,w \in Z$ there is $x = zg^n$ and $y = wg^m$. We now do some manipulation
\begin{align*}
xy &= zg^nwg^m \\
&= zwg^ng^m \\
&= wzg^mg^n \\
&= wg^mzg^n \\
&= yz
\end{align*}
and we find $G$ is abelian. This means $Z = Z(G) = G$ so $G/Z = (e)$.

\subsubsection{Question 12}
If $G/N$ is abelian then we get $Nab = Nba$. This means there is some $n,m \in N$ such that $nab = mba$, and it follows that $(nab)^{-1} = (mba)^{-1} = a^{-1}b^{-1}m^{-1}$. We now get that $e = nab(nab)^{-1} = naba^{-1}b^{-1}m^{-1}$ so $n^{-1}m = aba^{-1}b^{-1}$. It follows from $N$ being a subgroup of $G$ that $aba^{-1}b^{-1} \in N$.

\subsubsection{Question 13}
Let $a,b\in G$. We get $aba^{-1}b^{-1}\in N$, and therefore $\eqclass e = \eqclass{aba^{-1}b^{-1}} = \eqclass{ab}\eqclass{a^{-1}b^{-1}}$. It follows then that $\eqclass{ab} = \eqclass{a^{-1}b^{-1}}^{-1} = \eqclass{\left(a^{-1}b^{-1}\right)^{-1}} = \eqclass{ba}$ so $G / N$ is abelian.

\subsubsection{Question 18}
Choose $a,b\in T$ then $a^n = e$ and $b^m = e$ and therefore $(ab)^{nm} = a^{nm}b^{nm} = e$ so $ab \in T$. Choose $a \in T$ and we get $a^n = e$ so $a^{n-1}a = e$ and thus $a^{n-1} = a^{-1} \in T$. We have now shown that $T$ is a subgroup of $G$.

Now for any $\eqclass{a}\in G/T$ if $\eqclass{a}^n = \eqclass{e}$ for some $n > 0$ then we would get $Ta^n = T$ and further for some $u,v\in T$, $va^n = u$. Now we if we let $t =v^{-1}u$ then we get $a^n = t \in T$. By definition of $T$ we get $t^m = e$ for some $m > 1$ and thus $\left(a^n\right)^m = a^{nm} = e$ so $a \in T$. It follows then that $\eqclass{a} = \eqclass{e}$, so only the identity has a finite order in $G/T$.


\subsection{Section 7}
\subsubsection{Question 4}
We know that the kernel of a homomorphism is a normal subgroup, so let us construct $\varphi:G\to G_2$ as $\varphi(a,b) = b$. $\varphi$ is a homomorphism as $\varphi(a,b)\varphi(c,d) = ac = \varphi(ac,bd)$. Now we will show $\ker(\varphi) = \setbuilder{\pair{e_1}x}{x \in G_2}$.

\subsubsection{Question 5}
\subsubsection{Question 7}

\subsection{Section 8}
\subsubsection{Question 4}
\subsubsection{Question 6}
\subsubsection{Question 7}
\subsubsection{Question 8}
\subsubsection{Question 11}
\subsubsection{Question 12}

\subsection{Section 9}
\subsubsection{Question 1}
\subsubsection{Question 2}
\subsubsection{Question 3}

\end{document}

